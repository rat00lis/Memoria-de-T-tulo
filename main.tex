%%%%%%%%%%%%%%%%%%%%%%%%%%%%%%%%%%%%%%%%%%%%%%%%%%%%%%%%%%%%%%%%%%%%%%%%%%%%%%%
%%%  Plantilla para la elaboración del informe final de memorias de título  %%%
%%%  Adaptada por Oliver Diego Antonio Alarcón                              %%%
%%%%%%%%%%%%%%%%%%%%%%%%%%%%%%%%%%%%%%%%%%%%%%%%%%%%%%%%%%%%%%%%%%%%%%%%%%%%%%%

\documentclass[12pt,twoside]{report}
\usepackage[utf8]{inputenc}
\usepackage[spanish,es-tabla]{babel}
\usepackage[inner=2.5cm,outer=2cm,top=2.5cm,bottom=2cm]{geometry}
\usepackage{amsmath, amsfonts, amssymb}
\usepackage{graphicx, float}
\usepackage{subcaption}
\usepackage{booktabs}
\usepackage{longtable}
\usepackage{enumitem}
\usepackage{adjustbox}
\usepackage{makecell}
\usepackage{hyperref}
\usepackage{xcolor}
\usepackage{titlesec}
\usepackage{setspace}
\usepackage{svg}
\usepackage{float}
\usepackage{rotating}
\usepackage{todonotes}
\graphicspath{{images/}}

\setlength{\parindent}{12pt}
\titleformat{\chapter}
{\bfseries\large}{\chaptertitlename~\thechapter. }{0pt}{}
\titleformat{\section}
{\bfseries\large}{\thesection. }{0pt}{}

\begin{document}
\pagestyle{empty}

%%%%%%% Portada (Anexo C)
\begin{titlepage}
    \includegraphics[height=55pt]{escudoUdec}
    \begin{minipage}[b][60pt][c]{70ex}
    	\bfseries
        UNIVERSIDAD DE CONCEPCIÓN\\
        FACULTAD DE INGENIERÍA\\
        DEPARTAMENTO DE INGENIERÍA INFORMÁTICA\\
        Y CIENCIAS DE LA COMPUTACIÓN
    \end{minipage}
    \vspace{3cm}
    
    \begin{center}	
        \uppercase{\large\textbf{Uso De Estructuras De Datos Compactas Para Visualización De Series De Tiempo}}

        \vspace{1.5cm}
        POR
        \bigskip

        \textbf{Oliver Diego Antonio Brito Alarcón}

        \vspace{2cm}
        {\small Memoria de Título presentada a la Facultad de Ingeniería de la Universidad de Concepción para\\
        optar al título profesional de Ingeniero Civil Informático}
	
        \vspace{2cm}	
        \textbf{Profesores Guía} \\
        \bigskip
        
        José Fuentes \\
        Gonzalo Rojas


        \vfill
        Julio 2025

        Concepción (Chile)
        
        \bigskip
        \textcopyright{}
        2025 Oliver Diego Antonio Alarcón
	\end{center}
\end{titlepage}

%%%%%%%% Copyright
\textcolor{white}{.}
\vfill
\noindent\textcopyright{}
2025 Oliver Diego Antonio Alarcón\\
Se autoriza la reproducción total o parcial, con fines académicos, por cualquier medio o procedimiento, incluyendo la cita bibliográfica del documento.

%%%%% Dedicatoria
\clearpage
\onehalfspace
\vspace*{5cm}
\begin{flushright}
	A quienes el destino me dio por familia y a quienes la vida me permitió elegir como tal.
\end{flushright}

%%%%% Agradecimientos
\chapter*{Agradecimientos}
\thispagestyle{empty}
Esta Memoria de Título fue posible gracias al apoyo invaluable de mi madre, mi hermano y mi pareja, quienes estuvieron a mi lado y me permitieron concentrarme cuando más lo necesitaba.

A mis abuelos, gracias por su cariño y fortaleza. Y a mis amigues, gracias por su escucha, su compañía y por ayudarme a no perder la cordura durante esta etapa.
%%%%% Resumen
\clearpage
\pagestyle{plain}
\pagenumbering{roman} 
\chapter*{Resumen}
El creciente volumen de datos en aplicaciones modernas ha generado la necesidad de técnicas eficientes para su almacenamiento, análisis y visualización. Esta memoria explora el uso de estructuras de datos compactas —como \texttt{enc\_vector}, \texttt{vlc\_vector} y \texttt{dac\_vector} de la biblioteca \texttt{SDSL}— para representar series de tiempo de gran tamaño, combinándolas con algoritmos de \textit{downsampling} que permiten visualizar grandes volúmenes de información sin perder interpretabilidad. Se implementa una biblioteca en Python, \texttt{cv\_visualization}, que actúa como capa de compatibilidad entre estas estructuras y librerías de visualización como \texttt{Plotly}, \texttt{Altair} y \texttt{PyGal}.

Se diseñan experimentos para evaluar el impacto de estas técnicas en el uso de memoria durante la visualización (es decir, cuánta memoria RAM es necesario asignar en tiempo de ejecución), el tiempo de renderizado y el espacio de almacenamiento ocupado por los datos una vez reducidos. El análisis teórico y empírico incluye explicaciones detalladas de los esquemas de codificación utilizados, su complejidad de acceso y la comparación entre representaciones tradicionales y compactas.

Los resultados demuestran que, si bien el \textit{downsampling} mejora la velocidad de visualización, el uso de estructuras compactas puede sacrificar parcialmente esa velocidad —aunque sigue siendo menor que la necesaria para visualizar los datos originales—, y reduce en gran medida el espacio que ocupan los datos downsampleados en disco. Se mantiene la fidelidad visual de las gráficas, pero se identifican como limitaciones que algunas bibliotecas de visualización no son completamente compatibles con estas estructuras, ya que pueden descomprimir los datos de forma interna, provocando que el uso de memoria en tiempo de ejecución aumente significativamente y limite los beneficios esperados.

No quedan claros cuáles son los beneficios logrados con el trabajo, más allá de las limitaciones.

\chapter*{Abstract}
The growing volume of data in modern applications has created the need for efficient techniques for storage, analysis, and visualization. This thesis explores the use of compact data structures —such as \texttt{enc\_vector}, \texttt{vlc\_vector}, and \texttt{dac\_vector} from the \texttt{SDSL} library— to represent large time series, combining them with \textit{downsampling} algorithms that allow the visualization of massive datasets without losing interpretability. A Python library, \texttt{cv\_visualization}, was developed to serve as a compatibility layer between these structures and visualization libraries such as \texttt{Plotly}, \texttt{Altair}, and \texttt{PyGal}.

Experiments were designed to evaluate the impact of these techniques on memory usage during visualization (i.e., how much RAM needs to be allocated at runtime), rendering time, and the storage space occupied by the reduced datasets. Theoretical and empirical analysis includes detailed explanations of the encoding schemes used, their access complexity, and the comparison between traditional and compact representations.

The results show that while \textit{downsampling} improves visualization speed, the use of compact structures may partially sacrifice that speed—although it remains lower than rendering the original data—and significantly reduces the disk space occupied by downsampled datasets. Visual fidelity is preserved, but a notable limitation is that some visualization libraries are not fully compatible with these structures, as they may internally decompress the data, causing runtime memory usage to increase substantially and limiting the expected efficiency benefits.

%%%%% Índices
\tableofcontents
\listoftables
\listoffigures

%%%%% Cuerpo principal
\clearpage
\pagenumbering{arabic}

\section{Introducción}
\subsection{Contexto General}

Dar forma gráfica a la información es una herramienta valiosa, permitiendo que la misma sea más fácil y rápida de entender, comunicar e incluso aprender. 

Algo tan simple como analizar manualmente los registros de un sistema, lo cual puede requerir una cantidad considerable de tiempo para comprender qué ocurrió en un determinado día o a una hora específica, puede graficarse como una serie de tiempo y permitir que ese análisis se haga en cuestión de segundos.

Todo esto es posible gracias a la \textbf{visualización}, que para efectos de esta Memoria de Título podemos entender como el uso de representaciones visuales interactivas, asistidas por computadora, para ayudar a la cognición.~\cite{card1999readings}. 

A pesar de esta definición, que posiciona la visualización como una herramienta que sienta sus bases en la computación, sus orígenes no se alinean con la llegada de la tecnología moderna. El adagio ``Una imagen vale más que mil palabras", atribuida al dramaturgo y poeta Henrik Johan Ibsen, demuestra cómo esta idea se sienta en el imaginario colectivo como una verdad lógica. Incluso remontándonos a tiempos previos a nuestra era, las civilizaciones tempranas utilizaron distintos métodos para representar datos de forma visual, como el mapa estelar de Suzhou de la dinastía Song en China, que mostraba 1.434 estrellas agrupadas en 280 constelaciones~\cite{bonnetbidaud2009dunhuang}. Este y otros ejemplos históricos nos demuestran el deseo humano de centrar sus esfuerzos en simplificar información compleja en formas comprensibles y accesibles, utilizando representaciones visuales que potencian nuestra capacidad para observar patrones, relaciones y significados que de otro modo pasarían desapercibidos.

Este concepto es aún más prevalente en los tiempos que corren, ya que la llegada de las computadoras y el software ha revolucionado nuestra capacidad de manejar la información. Es gracias a estos avances tecnológicos que podemos analizar grandes volúmenes de información compleja, que normalmente no podríamos haber procesado sin la aparición de software específico para este objetivo. Nos referiremos a este tipo de datos como \textbf{Macrodatos} o \textbf{Big Data}\cite{mcafee2012bigdata}.

\subsubsection{Big Data}

La definición de \textit{Big Data} corresponde a la información de gran volumen, gran velocidad y gran variedad que requiere formas innovadoras y eficientes para su procesamiento. Sin embargo, este término se popularizó alrededor del año 2012~\cite{diebold2012bigdata}, y lo que se consideraba \textit{Big Data} en el pasado es de mucha menor magnitud de lo que se considera hoy en día. Así mismo, la magnitud de estos datos en un futuro será mayor, ya que las capacidades de almacenamiento y procesamiento aumentarán.

Por tanto, se considerarán algunas características esenciales para asegurarnos de que estamos lidiando con macrodatos.
\begin{enumerate}
    \item \textbf{Variedad}:         El conjunto de datos es heterogéneo, y normalmente no posee una estructura concreta para su procesamiento de manera nativa.
    \item \textbf{Velocidad}: 
        Se refiera a la rapidez con la que los datos se generan. Normalmente los tipos de datos a analizar están en tiempo real, y generan nuevos datos a una frecuencia alta.
    \item \textbf{Volumen}: 
        La cantidad de datos es significativa para el sistema donde se está trabajando y la tecnología actual.
\end{enumerate}

Otra característica importante a considerar de esta definición, es que estos datos son \textbf{inútiles} por sí solos,~\cite{gandomi2015beyond} la relevancia de este gran volúmen de información proviene del conjunto de \textbf{todos} los datos y la \textbf{interpretación} de los mismos. 

Al representar visualmente, por ejemplo, una serie de tiempo, es necesario acceder a cada uno de los elementos contenidos en el conjunto de datos para construir la gráfica correspondiente. Supongamos que disponemos de una secuencia de \( n \) puntos ordenados \((x_1, y_1), (x_2, y_2), \ldots, (x_n, y_n)\), los cuales deben ser procesados por el motor de renderizado de la herramienta de visualización. Este proceso conlleva, en el mejor de los casos, una complejidad temporal \(\mathcal{O}(n)\), ya que se requiere al menos una operación por cada punto para construir su representación gráfica.

Este comportamiento lineal implica que el tiempo necesario para renderizar una visualización crece proporcionalmente con la cantidad de datos. Por ejemplo, en la Figura~\ref{add_trace_plotly} se observa cómo la visualización de una serie de aproximadamente 350 mil puntos requiere cerca de 1.5 segundos para ser completada. Este tiempo, aunque aparentemente breve, resulta considerable en escenarios donde se espera interacción fluida, como el análisis exploratorio o los tableros de monitoreo en tiempo real.

\begin{figure}
    \centering
    \includegraphics[width=0.9\linewidth]{introduction/images/add_trace_plotly.png}
    \title{figure}
    \caption{Tiempo promedio en segundos de graficar N cantidad de puntos usando la librería plotly.}
    \label{add_trace_plotly}
\end{figure}

De acuerdo con el informe publicado por Visa en 2017~\cite{visa2017facts}, su plataforma global de pagos posee la capacidad de procesar hasta 65{,}000 transacciones por segundo en condiciones de carga máxima. Sin embargo, en escenarios de uso más comunes, el volumen de transacciones tiende a rondar las 2{,}000 por segundo. Si consideramos un análisis retrospectivo de un solo día completo, el volumen total de datos asciende a un intervalo entre aproximadamente \(2{,}000 \times 86{,}400 = 1.728 \times 10^8\) (uso bajo) y \(65{,}000 \times 86{,}400 = 5.616 \times 10^9\) (uso máximo) transacciones por día.

Bajo este escenario, un trabajador que requiera analizar la actividad completa de un solo día enfrentaría una carga de datos que oscila entre cientos de millones y varios miles de millones de registros. Si asumimos un proceso de visualización con complejidad temporal \(\mathcal{O}(n)\), incluso los enfoques más optimizados en términos de eficiencia enfrentarán tiempos de procesamiento significativos y haciendo inviable su análisis oportuno.

Además del rendimiento, cuando trabajamos con series temporales extremadamente densas o con puntos altamente consecutivos entre sí, obtenemos gráficos difíciles de interpretar visualmente. Este problema genera la impresión de gráficos comprimidos o saturados (``quashed"), dificultando enormemente su cognición, como puede observarse claramente en la Figura \ref{squashed_plot}, que representa la lectura de un sensor colocado en un puente.

El creciente volumen de datos en contextos de Big Data nos obliga a repensar los fundamentos mismos de la visualización. Si graficar cientos de miles de puntos ya representa un reto técnico y cognitivo, ¿qué ocurre cuando nos enfrentamos a millones o incluso miles de millones de registros? En estos casos, la visualización deja de ser una simple representación directa de los datos y se transforma en un proceso de síntesis y decisión: ¿qué se debe mostrar, qué se puede omitir, y cómo evitar distorsionar la información en el camino?


\subsubsection{Downsampling}

Mejorar la eficiencia de los procesos que permiten la visualización es una manera de reducir los tiempos de cómputo, pero existen millones de escenarios donde esto no es suficiente. Aún con estas mejoras, llega una cantidad de puntos que simplemente no es viable visualizar en un tiempo prudente. Así mismo, la legibilidad de los gráficos no se resuelve con estos avances.

Es complejo escapar de esta problemática, ya que no podemos reducir la complejidad temporal lineal a menos que \textbf{no grafiquemos todos los puntos}.

Bajo esa idea surge la técnica conocida como \textit{downsampling}, la cual consiste en seleccionar solo un subconjunto de los puntos originales para ser graficados. A diferencia de métodos que generan nuevos datos mediante agregaciones o promedios, el \textit{downsampling} busca preservar puntos existentes del conjunto original que sean representativos de su comportamiento general. Su objetivo no es reemplazar el análisis detallado, sino facilitar la interpretación visual de grandes volúmenes de datos en contextos donde mostrar todos los puntos no es viable ni útil~\cite{steinarsson2013downsampling}. Esta técnica permite reducir el tiempo de renderizado y evitar la saturación visual sin comprometer, en teoría, la percepción de patrones relevantes. No obstante, su aplicación exige un criterio claro: ¿qué puntos se deben conservar para que la visualización siga siendo fiel a los datos?

Este acercamiento permite mejorar el rendimiento y la legibilidad sin comprometer significativamente la percepción de patrones relevantes. Existen múltiples algoritmos diseñados con este fin, los cuales serán analizados en detalle en la Sección~\ref{alternatives}.

\squashedplot
\downsample

El grupo de investigación \textbf{PreDiCT.IDLab}\cite{predict2025}, afiliado a la Universidad de Gante en Bélgica, ha desarrollado e investigado activamente métodos avanzados para la visualización eficiente de series de tiempo. Entre sus contribuciones destaca la herramienta \texttt{tsdownsample}, un conjunto de algoritmos de alto rendimiento diseñados específicamente para realizar downsampling en contextos de visualización~\cite{tsdownsample}, la cual implementa los algoritmos ya mencionados para ser utilizados en \textit{Python}.

Otra biblioteca de gran utilidad creada por estos académicos es \textit{plotly-resampler}~\cite{plotly-resampler}, la cual integra \textit{tsdownsample} de manera directa en \textit{plotly}. A pesar de el preprocesamiento necesario para generar una submuestra de una serie de tiempo, aún así podemos ver diferencias significativas en el tiempo necesario para la visualización de conjuntos grandes de información.

Esta mejora es evidente en la Figura \ref{plotly_vs_resampler}, donde podemos observar cómo usar \textit{plotly-resampler} disminuye el tiempo de renderización del gráfico para un conjunto de pares ordenados, a pesar de que es un \textit{wrapper} de \textit{plotly} y procesa el vector original. Esto se debe a que la cantidad de puntos seleccionados $(n_{out})$ es constante, por lo que la complejidad de crear un gráfico con técnicas de \textit{downsample} es $\mathcal{O}(n_{out})$, o sea, $\mathcal{O}(1)$.

\plotlyresampler

En definitiva, usar estrategias de \textit{downsample} es una opción muy útil a la hora de visualizar una serie de tiempo con muchos puntos, ya que permite mejorar la legibilidad, tiempo de renderización y, a causa de esto, el análisis de los datos que se están estudiando.

\subsection{Presentación del Problema}

A pesar de todas las ventajas que ofrece el uso de técnicas de \textit{downsampling} —como la mejora en el tiempo de renderizado, la reducción de la sobrecarga cognitiva y una experiencia más fluida al visualizar datos densos—, estas no necesariamente abordan todos los aspectos posibles de optimización dentro del proceso de visualización de series de tiempo.

%se marca en un marco mas grande que la memoria se acerca pero es mas otrabajo q esta memoria de titulo -- uno no reemplaza al otro -- explicar la idea de tener datos originales en memoria junto a downsampleados

% downsample te ayuda a visualizar pero tmb es necesario operar sobre los datos originales
Por ejemplo, herramientas como \textit{tsdownsample} permiten reducir la cantidad de puntos a representar, generando así gráficos más livianos y veloces. Sin embargo, incluso los datos ya reducidos siguen representándose con estructuras estándar que pueden ocupar un espacio considerable en memoria. En otras palabras, se reduce la cantidad de información a procesar, pero no necesariamente el tamaño de esa información en términos de espacio ocupado.

Esto plantea una nueva dimensión del problema: si ya hemos logrado disminuir significativamente el tiempo requerido para visualizar datos, ¿por qué no aprovechar esa mejora para optimizar también el espacio que ocupan los datos resultantes? Es decir, en lugar de mantenernos solo en la reducción de puntos, podríamos buscar formas de representar esa información de manera más eficiente, reduciendo aún más el tamaño total ocupado sin perder la capacidad de analizarla visualmente.

Un enfoque inicial para abordar esta problemática consiste en emplear \textbf{estructuras de datos compactas} con el objetivo de reducir aún más el espacio requerido para almacenar los datos visualizados. No obstante, estas estructuras presentan limitaciones importantes: muchas de ellas no permiten representar números flotantes, negativos e incluso, en algunos casos, el valor cero, lo que restringe su aplicabilidad en escenarios donde los datos numéricos presentan mayor diversidad.

Esta inquietud conduce a la exploración de técnicas adicionales de representación que, incluso a costa de un ligero incremento en el tiempo de cómputo (que ya se ha reducido en su complejidad), permitan disminuir de manera significativa el espacio ocupado por los datos. El objetivo es integrar estas estrategias de forma complementaria con los métodos de \textit{downsampling} existentes, logrando así un equilibrio más eficiente entre el rendimiento temporal y la optimización espacial.



\section{Objetivos del Proyecto}

El objetivo general de esta memoria es investigar si el uso de estructuras de datos compactas permite optimizar la visualización de series de tiempo de alta densidad, considerando tanto el rendimiento como la eficiencia en el uso del espacio.

\subsection{Objetivos Específicos}
\begin{enumerate}
\item Desarrollar herramientas que permitan evaluar el rendimiento y el uso de memoria durante la visualización de conjuntos de datos de gran tamaño.
\item Analizar el equilibrio entre uso de espacio y rendimiento al emplear estructuras de datos compactas en distintas etapas del proceso de downsampling.
\end{enumerate}


\section{Evaluación de Alternativas}
\label{alternatives}

\subsection{Técnicas de Downsampling}

Una vez introducido el concepto de downsampling como solución al problema de visualización de series de tiempo muy densas, es necesario revisar y comparar las principales alternativas existentes.

Diversos autores han propuesto algoritmos que permiten seleccionar los puntos más representativos de una serie para facilitar su visualización. A continuación, se presenta un resumen de los métodos más relevantes:

\begin{enumerate}
    \item \textbf{Mode-Median-Bucket (MMB)}: 
    Divide los datos en bloques y selecciona un punto por bloque usando la moda o la mediana de los valores. Es simple y fácil de entender, pero tiende a omitir picos y valles locales relevantes~\cite{steinarsson2013downsampling}.

    \item \textbf{Min-Std-Error-Bucket (MSEB)}: 
    Utiliza regresión lineal para calcular el error estándar entre pares de puntos, eligiendo aquellos que minimizan la suma de errores. Produce resultados estadísticamente coherentes, pero suaviza demasiado la serie y elimina detalles visuales importantes~\cite{steinarsson2013downsampling}.

    \item \textbf{Longest-Line-Bucket (LLB)}: 
    Similar a MSEB, pero en lugar de minimizar el error, maximiza la longitud total de las líneas entre puntos seleccionados. Tiene mejor capacidad para conservar picos extremos y fluctuaciones relevantes~\cite{steinarsson2013downsampling}.

    \item \textbf{Largest-Triangle-Three-Buckets (LTTB)}: 
    Divide los datos en tres bloques consecutivos y selecciona el punto que forma el triángulo de mayor área con los puntos de los bloques adyacentes. Esta técnica preserva bien la forma general del gráfico y es eficiente computacionalmente~\cite{steinarsson2013downsampling}.

    \item \textbf{Largest-Triangle-Dynamic (LTD)}: 
    Variante del LTTB que adapta dinámicamente el tamaño de los bloques según la variación local de los datos. Mejora la representación visual en series con regiones muy fluctuantes y otras más estables~\cite{steinarsson2013downsampling}.

    \item \textbf{MinMaxLTTB}: 
    Propuesto por Van Der Donckt {\it et al.}~\cite{vanderdonckt2023minmaxlttb}, este algoritmo mejora la escalabilidad de LTTB al aplicar una preselección eficiente de puntos mínimos y máximos verticales mediante el algoritmo MinMax, para luego aplicar LTTB solo sobre esos puntos seleccionados. De esta manera, reduce significativamente el tiempo de cómputo sin sacrificar calidad visual.
\end{enumerate}

Dado el enfoque de este trabajo, se seleccionó principalmente el uso del algoritmo \textbf{MinMaxLTTB}. Esta decisión se fundamenta en varios factores: en primer lugar, el algoritmo ha sido propuesto por los mismos autores que desarrollaron la biblioteca \texttt{tsdownsample}, lo que garantiza una implementación de referencia optimizada y bien documentada. Además, MinMaxLTTB presenta un excelente equilibrio entre rendimiento y fidelidad visual, al combinar una reducción sustancial en el tiempo de cómputo con una buena preservación de la forma general de la serie de tiempo.

Como complemento para el análisis comparativo, también se utilizará el algoritmo \textbf{LTTB}, ya que es ampliamente citado en la literatura como una opción base eficiente, y su comparación directa con MinMaxLTTB permite observar el impacto de la estrategia de preselección aplicada por los autores.

\subsection{Estructuras de Datos Compactas}

El uso de estructuras de datos compactas permite representar secuencias numéricas en memoria utilizando una menor cantidad de bits por elemento, manteniendo el acceso a los datos sin pérdida de información. 

En la literatura existen bibliotecas que implementan estas estructuras de datos compactas. En particular, utilizaremos la biblioteca \texttt{sdsl4py}, un conjunto de bindings en Python para la Succinct Data Structure Library (SDSL), que permite acceder a múltiples representaciones compactas desde  Python. A continuación se presentan las  estructuras de datos compactas que fueron consideradas en esta memoria de título:

\begin{enumerate}
    \item \textbf{enc\_vector\_elias\_gamma}:
    Utiliza codificación de enteros Elias-Gamma. Es eficiente para valores pequeños, ya que su tamaño codificado crece logarítmicamente con el valor. No puede representar ceros.

    \item \textbf{enc\_vector\_elias\_delta}:
    Variante de Elias-Gamma que mejora la compresión para valores grandes, manteniendo eficiencia en lectura. También requiere que todos los valores sean mayores que cero.

    \item \textbf{enc\_vector\_fibonacci}:
    Usa codificación basada en la sucesión de Fibonacci. Esta técnica es compacta y libre de prefijos, pero tiene limitaciones para valores cero y una complejidad levemente mayor en decodificación.

    \item \textbf{enc\_vector\_comma\_2}:
    Codificación por comas binarias (comma-2), una forma simple y eficiente de codificar enteros con separación de bits mediante una secuencia especial. Ofrece buena compresión para secuencias no muy dispersas.

    \item \textbf{vlc\_vector\_elias\_delta}:
    Variante del enc\_vector\_elias\_delta que permite acceso más rápido a los elementos gracias a una estructura de lectura optimizada (Variable-Length Codes con índice de acceso directo).

    \item \textbf{vlc\_vector\_elias\_gamma}:
    Equivalente a la anterior, pero usando codificación Elias-Gamma. Útil cuando se prioriza el acceso sobre la compresión máxima.

    \item \textbf{vlc\_vector\_fibonacci}:
    Combina las ventajas de codificación Fibonacci con accesos más eficientes mediante índices auxiliares.

    \item \textbf{vlc\_vector\_comma\_2}:
    Aplica codificación comma-2 con estructura de acceso rápido. Es útil cuando se necesita acceder a datos comprimidos con latencias bajas.

    \item \textbf{dac\_vector}:
    Utiliza Directly Addressable Codes (DAC), una estructura escalonada que permite un buen balance entre compresión y velocidad de acceso. Es especialmente útil para secuencias de enteros donde los valores tienen alta varianza.

\end{enumerate}

Para evaluar estas estructuras de datos, se han medido tres métricas clave: el tiempo de acceso a los elementos, el tiempo de construcción de la estructura y el espacio utilizado en memoria. 

\paragraph{Input}
\vspace{0.2cm}

Para garantizar una evaluación comprehensiva y representativa del comportamiento de las estructuras de datos comprimidas, se han diseñado cuatro tipos de vectores sintéticos que simulan diferentes patrones de datos comúnmente encontrados en aplicaciones reales:

\begin{enumerate}
    \item \textbf{Incremental pequeño (\texttt{incremental\_small}):} Vectores con valores crecientes que presentan pequeñas diferencias entre elementos consecutivos. Este patrón simula series temporales con crecimiento gradual o datos de sensores con variaciones mínimas. Los valores de $x$ siguen la secuencia $[1, 2, 3, ..., n]$, mientras que los valores de $y$ añaden una pequeña variación aleatoria en el rango $[0, 2]$ a cada posición.

    \item \textbf{Aleatorio (\texttt{random}):} Vectores con valores completamente aleatorios distribuidos uniformemente en un rango amplio. Este patrón representa el peor caso para algoritmos de compresión, ya que no presenta patrones predecibles. Tanto $x$ como $y$ contienen enteros aleatorios en el rango $[1, n \times 10]$.

    \item \textbf{Oscilante (\texttt{oscillating}):} Vectores que siguen un patrón de onda sinusoidal, alternando entre valores ascendentes y descendentes. Este comportamiento es típico en señales periódicas, datos de vibración o patrones cíclicos. Los valores de $x$ son secuenciales, mientras que $y$ sigue la función $y_i = \frac{n}{2}(1 + \sin(\frac{4\pi i}{n}))$.

    \item \textbf{Incremental grande (\texttt{incremental\_large}):} Vectores con valores crecientes que presentan grandes diferencias entre elementos consecutivos. Este patrón simula datos con crecimiento exponencial o escalas logarítmicas. Los valores de $x$ son secuenciales $[1, 2, 3, ..., n]$, y los valores de $y$ se calculan como $y_i = i \times \text{random}[50, 200]$.
\end{enumerate}

Cada vector de prueba contiene por defecto 10,000 elementos, proporcionando un volumen de datos suficiente para evaluar el rendimiento de las estructuras de datos compactas. Esta diversidad de patrones permite analizar cómo cada método de compresión se adapta a diferentes características de los datos, desde el caso ideal (datos con alta correlación) hasta el caso más desafiante (datos completamente aleatorios).

\paragraph{Resultados}
\vspace{0.2cm}

A continuación, se presenta un promedio de los resultados obtenidos al evaluar las distintas estructuras de datos compactas utilizando los vectores sintéticos mencionados anteriormente.

% añadir imagenes y tablas de resultados
\begin{figure}[H]
\centering
\includegraphics[width=0.8\textwidth]{alternatives/scatter_plot/3d_scatter_plot_structures.png}
\caption{Puntajes promedio de las estructuras de datos compactas}   
\label{fig:scatter_plot_structures}
\end{figure}

\begin{figure}[H]
\centering
\includegraphics[width=0.8\textwidth]{alternatives/scatter_plot/euclidian_distance_from_ideal.png}
\caption{Distancia euclidiana promedio de las estructuras de datos compactas respecto al punto $(0, 0, 0)$}   
\label{fig:euclidian_distance_from_ideal}
\end{figure}

\input{alternatives/scatter_plot/latex_ranking_table.tex}

% Como podemos apreciar en el Anexo~\ref{anexo_sdsl4py}, al evaluar la velocidad de acceso, el tiempo de construcción y el espacio utilizado por las distintas alternativas de estructuras de datos compactas, no se observa una diferencia significativa en cuanto a su curva de complejidad teórica. No obstante, para determinar cuál estructura resulta más adecuada en la práctica, se define una \textbf{relación compuesta} que pondera las tres métricas principales.

% Esta relación se formula como:
% %scatter plot para seleccionar la estructura de datos compacta 3d

% \begin{equation}
% \text{Score}(E) = \alpha \cdot \text{AccessTime}(E) + \beta \cdot \text{BuildTime}(E) + \gamma \cdot \text{SpaceUsed}(E)
% \end{equation}

% donde:
% \begin{itemize}
%     \item $E$ representa una estructura evaluada,
%     \item $\text{AccessTime}(E)$, $\text{BuildTime}(E)$ y $\text{SpaceUsed}(E)$ corresponden a los valores empíricos medidos para dicha estructura,
%     \item y $\alpha$, $\beta$, y $\gamma$ son coeficientes de ponderación tales que $\alpha + \beta + \gamma = 1$.
% \end{itemize}

% Estos coeficientes pueden ser ajustados en función de los objetivos específicos del sistema o aplicación. Por ejemplo, en un entorno donde el rendimiento en lectura sea prioritario, se puede asignar un valor mayor a $\alpha$; en cambio, si se trabaja con recursos de almacenamiento limitados, se puede incrementar el peso de $\gamma$. 

% En esta ocasión, se consideran los valores:

% \begin{itemize}
%     \item $\alpha = 0.6$
% 	\item $\beta = 0.4$
% 	\item $\gamma = 0.3$
% \end{itemize}

% \begin{table}[H]
% \centering
% \caption{Puntaje promedio por estructura (menor es mejor)}
% \label{tab:score_promedio_estructuras}
% \begin{tabular}{l r}
% \toprule 
% \textbf{Estructura} & \textbf{Score promedio} \\
% \midrule
% dac\_vector - 8              & 65899.74 \\
% vlc\_vector\_fibonacci - 8   & 69255.21 \\
% vlc\_vector\_elias\_delta - 8 & 73931.74 \\
% vlc\_vector\_elias\_gamma - 8 & 78023.47 \\
% vlc\_vector\_comma\_2 - 8     & 80850.67 \\
% Original Data                & 657595.20 \\
% \bottomrule
% \end{tabular}
% \end{table}

A partir de la puntuación resultante, se seleccionan las estructuras de datos \textit{dac\_vector} y \textit{vlc\_vector\_fibonacci} como las más adecuadas para el sistema, ya que son las que obtuvieron los puntajes más favorables en la evaluación.
\subsection{Framework de Experimentación}

%aqui falta hilar un poco estas ideas...
Benchmarking es el proceso continuo de comparar el desempeño, productos o prácticas de una organización, sistema o componente con referentes externos —como líderes de la industria o soluciones de alto rendimiento— con el objetivo de identificar brechas y oportunidades de mejora. Este proceso no se limita sólo a la recolección de métricas, sino que también implica el análisis crítico y, cuando es pertinente, la adopción o adaptación de ideas y métodos que han demostrado ser eficaces.~\ref{stapenhurst2009benchmarking}

\subsubsection{Benchmarking aplicado a esta memoria}

En el contexto de esta memoria, el benchmarking se utiliza como herramienta para evaluar el desempeño de distintas estrategias de visualización y almacenamiento de datos comprimidos, comparando tanto métricas de eficiencia (uso de memoria, tiempo de reconstrucción, velocidad de acceso) como la calidad de visualización obtenida tras la compresión. El objetivo es identificar si la solución propuesta (...) proporciona una ventaja en su uso respecto a las herramientas ya existentes explicadas en la sección de Evaluación de Alternativas.

\subsubsection{Herramientas de benchmarking}

Para poder comparar vamos a usar sacredd, que es una libreria de python Sacred is a tool to configure, organize, log and reproduce computational experiments. It is designed to introduce only minimal overhead, while encouraging modularity and configurability of experiments.~\ref{greff2017sacred}

Gracias a esta herramienta, podemos ejecutar experimentos de manera modular y con control de los parámetros. Se definen parametros globales dentro de la configuracion estandar que se establecio, pero podemos escribir configuracion especifica para cada experimento que hará un override de las configuraciones globales.

Por ejemplo, en las configuraciones globales, definimos iterations como 100, para asegurarnos de que cada instancia de un experimento es ejecutada 100 veces y se calcule un promedio de dichas ejecuciones. Sin embargo, para experimentos especificos definimos una configuracion local para hacer un override de esta configuracion. Asi, por default, todos los exp usan 100 iteraciones, pero los exp de memoria solo usan 1 iteracion ya que no tenemos factores externos que entorpezcan y se necesite un promedio como si sucede con los experimentos de tiempo.

Con la ayuda de dicha librería, se declaran las siguientes herramientas:

\paragraph{Input Handler}
\vspace{0.5em}
\textit{InputHandler} es una clase auxiliar que permite cargar datos desde archivos CSV y convertirlos en diferentes formatos, ya sea sin compresión, con compresión usando \textit{CompressedVector}, o bien aplicando técnicas de reducción de puntos. Su diseño parametrizable permite ajustar el formato de los datos para facilitar su uso en pruebas de benchmarking.

\vspace{0.5em}
\textbf{Configuración:} mediante el método \textit{set\_width()}, se puede definir el ancho de bits para la parte entera de los datos en los ejes \textit{x} e \textit{y}, lo que incide directamente en la eficiencia del almacenamiento comprimido.

\vspace{0.5em}
\textbf{Extracción de datos:} la función \textit{get\_from\_file()} permite cargar columnas desde archivos CSV, entregando los datos en distintos formatos según el parámetro \textit{option}:
\begin{itemize}
    \item \textit{"default"}: datos crudos en \textit{float}.
    \item \textit{"compressed\_vector"}: representación comprimida con \textit{CompressedVector}.
    \item \textit{"compressed\_vector\_downsampler"}: compresión + reducción usando \textit{CompressedVectorDownsampler}.
    \item \textit{"tsdownsample"}: reducción usando librería externa \textit{tsdownsample}.
\end{itemize}

\vspace{0.5em}
\textbf{Compresión:} el método interno \textit{compress\_vector()} se encarga de transformar una lista de valores en un \textit{CompressedVector}, aplicando las opciones de redondeo, compresión, y configuración de bits especificadas por el usuario.

\vspace{0.5em}
\textbf{Índices seleccionados:} luego de aplicar reducción, es posible recuperar los índices de los puntos seleccionados mediante \textit{get\_x\_indices()} y \textit{get\_y\_indices()}, lo que resulta útil para visualizar los puntos elegidos o analizar su distribución.

\paragraph{Experiment Runner}
\vspace{0.5em}

Llamamos \textit{Experiment Runner} a una serie de herramientas definidas en el benchmarking que nos permiten ejecutar cualquier experimento previamente definido. Este módulo gestiona las configuraciones globales, aplica los parámetros correspondientes a cada caso, ejecuta la función de evaluación y almacena los resultados mediante la librería \textit{Sacred} en un archivo \textit{.json}.

Por cada experimento, se registra la siguiente información:

\begin{itemize}
    \item \textit{option}: tipo de representación utilizada (ej. \textit{default}, \textit{compressed\_vector}, etc.).
    \item \textit{file}: nombre base del archivo de entrada.
    \item \textit{n\_size}: cantidad de datos utilizados en la muestra.
    \item \textit{n\_out}: tamaño de la salida en caso de reducción.
    \item \textit{mean}: promedio del valor medido en las iteraciones.
    \item \textit{stdev}: desviación estándar del valor medido.
    \item \textit{min} y \textit{max}: valores mínimo y máximo observados.
    \item \textit{all\_differences}: lista completa con los resultados individuales por iteración.
    \item \textit{iterations}: cantidad de veces que se repitió el experimento.
    \item \textit{measurement\_unit}: unidad de medida correspondiente (por ejemplo, milisegundos o bytes).
\end{itemize}

Estos datos permiten evaluar de forma cuantitativa el comportamiento de diferentes estrategias de compresión y visualización frente a distintos volúmenes de entrada.

Además de los resultados cuantitativos de cada experimento, el archivo \textit{.json} generado por \textit{Sacred} incluye información complementaria útil para trazabilidad y reproducibilidad. Por ejemplo, se registra el nombre del experimento, la ruta base de ejecución, las versiones exactas de las dependencias utilizadas, los archivos fuente involucrados, y metadatos del sistema donde se ejecutó (CPU, sistema operativo, versión de Python, etc.). También se incluyen los commits y estado del repositorio \textit{git}, permitiendo asociar los resultados directamente con una versión específica del código. Esta información permite contextualizar los resultados y asegurar que los experimentos puedan ser replicados en el futuro con fidelidad.

\paragraph{Definición de Experimentos}
\vspace{0.5em}

Dentro del repositorio, el archivo \textit{template.py} actúa como base para definir cualquier experimento compatible con el framework de benchmarking. Este archivo contiene una estructura mínima y parametrizable que permite registrar, configurar y ejecutar un experimento completo con la ayuda de \textit{Sacred}.

La estructura del experimento tiene tres componentes clave:

\begin{itemize}
    \item \textbf{Título del experimento:} definido mediante \textit{exp\_name}, permite organizar los resultados en carpetas separadas y legibles.
    \item \textbf{Configuración:} a través del decorador \textit{@exp.config} se definen los \textit{casos de prueba} (por ejemplo, vectores sin comprimir, comprimidos, o reducidos), así como el rango de tamaños de entrada y otros parámetros globales.
    \item \textbf{Función principal:} bajo el decorador \textit{@exp.automain}, se implementa la función que será ejecutada con los parámetros definidos. Esta instancia recibe una función \textit{experiment\_fn}, que representa la lógica específica del experimento (por ejemplo, medir tiempo de ejecución), y se conecta con el \textit{Experiment Runner} para la ejecución repetida, registro de métricas y almacenamiento de resultados.
\end{itemize}

Dentro de \textit{experiment\_fn(x, y, option)}, el usuario define lo que desea medir utilizando los datos entregados por el \textit{InputHandler}. Por ejemplo, se puede medir el tiempo de ejecución de un proceso sobre los datos \textit{x} e \textit{y}, o bien calcular la precisión de una reconstrucción. Este valor es luego devuelto al \textit{Experiment Runner}, que lo almacena junto con las configuraciones usadas.

Gracias a esta plantilla, definir nuevos experimentos se vuelve un proceso sencillo, flexible y reutilizable, lo que permite escalar y mantener la calidad del benchmarking durante el desarrollo.


\subsection{Uso de sdsl4py para la visualizacion de datos}

A pesar de que podemos utilizar el int\_vector de sdsl4py para pasarlo directamente a herramientas de visualizacion como por ej plotly, tenemos la limitacion de no poder usar numeros flotantes.

originalmente la libreria sdsl4py esta basada en sdsl-lite. Las estructuras de datos compactas no estan pensadas para usar otra cosa que no sean unsigned ints, ya que se basan principalmente en la repeticion de sus valores para la compresion, y al introducir decimales esto se complejiza porque los decimales de nummeros en  porr ej una serie de tiempo no necesariamente siguen el mismo patron que los valores originales, por no mencionar que pueden no presentar patrones en lo absoluto, por lo que se complejiza de enorme manera la representacion de numeros negativos y flotantes en int\_vectors y por tanto tambien en las diferentes formas de compresion de estos (antes de esto entonces hay que explicar como funciona la libreria brevemente, onda mencionar que primero se crea un int\_vector que es modifiucable, que hay que establecerr el tamanio del vector, que despues de aplicar una estrategia de compresion ya no es editable, etc etc).

\subsection{Compressed Vector}

Para superar la limitación de que \textit{sdsl4py} sólo permite almacenar enteros sin signo (\textit{uint}), diseñamos una estructura auxiliar llamada \textit{CompressedVector}, la cual permite representar números flotantes (positivos y negativos) utilizando múltiples vectores \textit{int\_vector<>} comprimidos. La idea principal consiste en descomponer cada número flotante en tres componentes:

\begin{enumerate}
    \item \textbf{Parte entera}: el valor absoluto truncado del número.
    \item \textbf{Parte decimal}: los dígitos decimales escalados como enteros.
    \item \textbf{Indicador de signo}: una codificación entera que indica si el valor es positivo, negativo o \textit{NaN}.
\end{enumerate}

Formalmente, dado un número real $x \in \mathbb{R}$, con una precisión deseada de $d$ decimales, se realiza la siguiente transformación:

\begin{align*}
    x &= \pm \left( \lfloor |x| \rfloor + \frac{\mathrm{decimales}(x)}{10^d} \right) \\
    \text{parte\_entera} &= \lfloor |x| \rfloor \\
    \text{parte\_decimal} &= \left\lfloor (|x| - \lfloor |x| \rfloor) \cdot 10^d \right\rfloor \\
    \text{signo} &=
        \begin{cases}
            1 & \text{si } x \geq 0 \\
            0 & \text{si } x < 0 \\
            2 & \text{si } x \text{ es } \textit{NaN}
        \end{cases}
\end{align*}

Cada uno de estos tres valores se almacena en un \textit{int\_vector<>} independiente, lo cual permite aplicar sobre ellos técnicas de compresión tradicionales provistas por \textit{sdsl4py} como \textit{bit\_compressed\_vector<>} o \textit{rrr\_vector<>}.

\vspace{1em}
\noindent
Para recuperar el número original en el índice $i$, se realiza la operación inversa:

\begin{align*}
    \text{si } \text{signo}[i] = 2, \quad &\Rightarrow x_i = \textit{NaN} \\
    \text{sino: } \quad &x_i = (-1)^{1 - \text{signo}[i]} \cdot \left( \text{parte\_entera}[i] + \frac{\text{parte\_decimal}[i]}{10^d} \right)
\end{align*}

\noindent
Este procedimiento está implementado en el método \textit{\_reconstruct\_float\_value} del código, y se basa en aritmética decimal precisa usando la clase \textit{Decimal} de Python para evitar errores de redondeo durante la reconstrucción.

\vspace{1em}
\noindent
Aunque esta solución utiliza más espacio que un único vector comprimido, permite una representación más expresiva y flexible de datos reales, conservando la capacidad de compresión eficiente de enteros provista por \textit{sdsl4py}.

Adicionalmente, para una correcta integración de \textbf{CompressedVector} con próximas librerías a ser utilizadas, así como las propias funciones nativas de Python con sus listas propias, se implementan distintas funciones que permiten operar por sobre la clase. En la siguiente tabla se resumen las operaciones disponibles, junto con la función mágica (\textit{dunder method}) que las implementa:


\begin{table}[H]
\centering
\renewcommand{\arraystretch}{1.3} % Aumenta la altura de las filas de forma consistente
\begin{tabular}{|p{3.5cm}|p{6cm}|p{5cm}|}
\hline
\textbf{Operación} & \textbf{Descripción} & \textbf{Método implementado} \\
\hline
\rule{0pt}{1.5em}Indexación         & Acceso a un elemento por índice                & \texttt{\_\_getitem\_\_(index)} \\
\hline
\rule{0pt}{1.5em}Asignación         & Modificación de un valor en un índice dado     & \texttt{\_\_setitem\_\_(index, value)} \\
\hline
\rule{0pt}{1.5em}Tamaño             & Cantidad de elementos almacenados              & \texttt{\_\_len\_\_()} \\
\hline
\rule{0pt}{1.5em}Iteración          & Permite iterar sobre los elementos             & \texttt{\_\_iter\_\_()} \\
\hline
\rule{0pt}{1.5em}Representación     & Muestra formato legible para impresión         & \texttt{\_\_repr\_\_()} \\
\hline
\rule{0pt}{1.5em}Suma               & Suma escalar a todos los elementos             & \texttt{\_\_add\_\_(other)} \\
\hline
\rule{0pt}{1.5em}Suma in-place      & Suma escalar modificando el vector             & \texttt{\_\_iadd\_\_(other)} \\
\hline
\rule{0pt}{1.5em}Resta              & Resta escalar a todos los elementos            & \texttt{\_\_sub\_\_(other)} \\
\hline
\rule{0pt}{1.5em}Resta in-place     & Resta escalar modificando el vector            & \texttt{\_\_isub\_\_(other)} \\
\hline
\rule{0pt}{1.5em}Multiplicación     & Multiplica todos los valores por escalar       & \texttt{\_\_mul\_\_(other)} \\
\hline
\rule{0pt}{1.5em}Multiplicación in-place & Multiplica modificando el vector         & \texttt{\_\_imul\_\_(other)} \\
\hline
\rule{0pt}{1.5em}División           & Divide todos los elementos por un escalar      & \texttt{\_\_truediv\_\_(other)} \\
\hline
\rule{0pt}{1.5em}División in-place  & Divide modificando el vector                   & \texttt{\_\_itruediv\_\_(other)} \\
\hline
\rule{0pt}{1.5em}Comparación        & Compara igualdad elemento a elemento           & \texttt{\_\_eq\_\_(other)} \\
\hline
\end{tabular}
\caption{Operaciones implementadas en la clase \texttt{CompressedVector}}
\end{table}

\subsection{CompressedVectorDownsampler}

Hasta el momento hemos hablado de estructuras de datos compactas por un lado, y por el otro de downsampling. La clase CompressedVectorDownsampler es una fusión de estos dos conceptos, entregándonos la información a graficar luego de pasar ambos procesos.

La clase integra CompressedVector y TsDownsample para dicho objetivo. El usuario pasa como parámetro x, y o ambos vectores de sus valores originales, así como la cantidad de decimales a considerar, el tamaño del entero a usar (entre 8,16,32 y 64),el metodo de compresion y de downsampling tanto como funciones o como strings y por ultimo los puntos de salida.

Con esto, se devuelve x, y o ambos vectores con el dowsampling ya aplicado, y con la estrategia de compresion tambien, de manera que el usuario puede almacenar un vector de gran densidad en una fración del espacio, garantizando menor uso de espacio y tiempo de renderización a la hora de graficar.

%aqui agregar imagen de funcionamiento

\subsection{Librerías de Visualización}

El objetivo principal de esta memoria es poder utilizar las herramientas generadas para visualizar la información luego de los procesos. Para efectos de la misma no se contempla la creación de una nueva herramienta de visualización, si no la adaptación y pruebas de bibliotecas ya existentes para dicho propósito.

A continuación, se hace un análisis de las bibliotecas analizadas para este trabajo.

\subsubsection{Requieren Numpy}

Esta seccion contiene todos los q requieren numpy y tienen el problema.

Existen numerosas bibliotecas para la visualización de datos. En este texto, ya se han mencionado textit{plotly} y textit{plotly-resampler} como ejemplo de ellas, aunque otra biblioteca muy popular y llena de funcionalidades es textit{matplotlib}~\cite{matplotlib}. Todas estas herramientas permiten renderizar una gran variedad de gráficos, ofrecen interactividad y más funciones que permiten un mejor análisis y cognibilidad de los datos.

Sin embargo, las bibliotecas mencionadas (y la gran mayoría de las ofertadas para python) tienen un problema (o tal vez feature) para ser usadas con CompressedVector. Esperan que el input sea un vector numpy o, de lo contrario, intentan transformar el vector input a numpy.

Esto sugiere un gran problema, ya que numpy se caracteriza por ser una biblioteca que espera un valor fijo del ancho del elemento en el vector, dentro de los dtypes ofrecidos por la misma biblioteca.

Por ende, esto va en contra de los vectores CompressedVector, que precisamente su objetivo es reducir la memoria de los elementos usados, y posee una variabilidad respecto al espacio (bits) de los elementos que contiene. 

Por ende, graficar con estas bibliotecas no es posible de primera mano. Para poder visualizar nuestro downsample comprimido debemos obligatoriamente transformarlo a numpy, para lo cual podemos setear la configuracion get\_decompressed como True. Esto sin embargo defeats the purpose ya que estaremos descomprimiendo los valores antes de usarlo, agregando una cantidad de memoria alocada mucho mayor y, al final, no usando realmente los valores comprimidos. Por ende solo tomaremos ventaja del downsample pero no de la compresion.
%aqui va el vector q compara alocation

La solución a este problema escapa del scope de esta memoria, ya que se tendria que cambiar el codigo base de estas librerias para poder iterar sobre un objeto iterable sin tener que traducirlo a numpy, o de lo contrario modificar numpy para poder integrar un nuevo dtype que funcione para CompressedVector. Ambos acercamientos al problema son demasiado time consuming y supondrian un gran esfuerzo y conocimiento para llevarse a cabo.

\subsubsection{PyGal}

PyGal es una biblioteca de python que está hecha para crear gráficos en formato svg para ser embbeded en paginas web~\cite{pygal}. Puede hacer graficos de linea, scatter, etc etc con bastantes funcinalidades. PyGal puede recibir cualquier objeto iterable sin la necesidad de ser convertido a numpy, por ello es compatible con CompressedVector y CompressedVectorDownsampler.

PyGal es similar al resto de bibliotecas que usan numpy, pero posee muchas limitaciones debido a su uso objetivo. Estas limitaciones abarcan desde sus formatos de salida hasta su nivel de interactividad, ya que no ofrece zoom ni seleccion, herramientas muy valiosas para el análisis de datos.

\subsubsection{Vega-Altair}

Vega-Altair is a declarative visualization library for Python. Its simple, friendly and consistent API, built on top of the powerful Vega-Lite grammar, empowers you to spend less time writing code and more time exploring your data~\cite{altair}.
Altair seria la biblioteca que mejor combina las ventajas de las dos opciones mencionadas anteriormente. Tiene la mayoria de funcionalidades de plotly y matplolib, y a la vez es compatible con CompressedVectorDownsampler sin la necesidad de transformarlo a ningun otro formato. Por ello, es la biblioteca donde mas se realizan experimentos para esta memoria, y la cual se recomienda su uso.


% All Libraries Memory Allocation
\DeclareRobustCommand{\AllLibrariesMemoryAllocationOnePlotLine}{
    %insertar imagen
    \begin{figure}[H]
        \centering
        \includegraphics[width=1\textwidth]{anexo/exp/All Libraries Memory Allocation/plots/All Libraries Memory Allocation_d_08_1_1_2_linear_line.png}
        \caption[]{Gráfico de memoria asignada por las diferentes bibliotecas al crear un gráfico para el input \textbf{d\_08\_1\_1\_2}.}
        \label{fig:all_libraries_memory_allocation_plot_line_1}
    \end{figure}
}

\DeclareRobustCommand{\AllLibrariesMemoryAllocationOnePlotBar}{
    %insertar imagen
    \begin{figure}[H]
        \centering
        \includegraphics[width=1\textwidth]{anexo/exp/All Libraries Memory Allocation/bar_plots/All Libraries Memory Allocation_d_08_1_1_2_log_bar.png}
        \caption[]{Gráfico de memoria asignada por las diferentes bibliotecas al crear un gráfico para el input \textbf{d\_08\_1\_1\_2}.}
        \label{fig:all_libraries_memory_allocation_plot_bar_1}
    \end{figure}
}

\DeclareRobustCommand{\AllLibrariesMemoryAllocationTwoPlotLine}{
    %insertar imagen
    \begin{figure}[H]
        \centering
        \includegraphics[width=1\textwidth]{anexo/exp/All Libraries Memory Allocation/plots/All Libraries Memory Allocation_d_08_1_1_10_linear_line.png}
        \caption[]{Gráfico de memoria asignada por las diferentes bibliotecas al crear un gráfico para el input \textbf{d\_08\_1\_1\_10}.}
        \label{fig:all_libraries_memory_allocation_plot_line_2}
    \end{figure}
}

\DeclareRobustCommand{\AllLibrariesMemoryAllocationTwoPlotBar}{
    %insertar imagen
    \begin{figure}[H]
        \centering
        \includegraphics[width=1\textwidth]{anexo/exp/All Libraries Memory Allocation/bar_plots/All Libraries Memory Allocation_d_08_1_1_10_log_bar.png}
        \caption[]{Gráfico de memoria asignada por las diferentes bibliotecas al crear un gráfico para el input \textbf{d\_08\_1\_1\_10}.}
        \label{fig:all_libraries_memory_allocation_plot_bar_2}
    \end{figure}
}

\DeclareRobustCommand{\AllLibrariesMemoryAllocationThreePlotLine}{
    %insertar imagen
    \begin{figure}[H]
        \centering
        \includegraphics[width=1\textwidth]{anexo/exp/All Libraries Memory Allocation/plots/All Libraries Memory Allocation_yellow_tripdata_2015-01_linear_line.png}
        \caption[]{Gráfico de memoria asignada por las diferentes bibliotecas al crear un gráfico para el input \textbf{yellow\_tripdata\_2015\_01}.}
        \label{fig:all_libraries_memory_allocation_plot_line_3}
    \end{figure}
}

\DeclareRobustCommand{\AllLibrariesMemoryAllocationThreePlotBar}{
    %insertar imagen
    \begin{figure}[H]
        \centering
        \includegraphics[width=1\textwidth]{anexo/exp/All Libraries Memory Allocation/bar_plots/All Libraries Memory Allocation_yellow_tripdata_2015-01_log_bar.png}
        \caption[]{Gráfico de memoria asignada por las diferentes bibliotecas al crear un gráfico para el input \textbf{yellow\_tripdata\_2015\_01}.}
        \label{fig:all_libraries_memory_allocation_plot_bar_3}
    \end{figure}
}




% All Libraries Time Comparison
\DeclareRobustCommand{\AllLibrariesTimeComparisonOnePlotLine}{
    %insertar imagen
    \begin{figure}[H]
        \centering
        \includegraphics[width=1\textwidth]{anexo/exp/All Libraries Time Comparison/plots/All Libraries Time Comparison_d_08_1_1_2_linear_line.png}
        \caption[]{Gráfico de tiempo de ejecución de las diferentes bibliotecas al crear un gráfico para el input \textbf{d\_08\_1\_1\_2}.}
        \label{fig:all_libraries_time_comparison_plot_line_1}
    \end{figure}
}

\DeclareRobustCommand{\AllLibrariesTimeComparisonOnePlotBar}{
    %insertar imagen
    \begin{figure}[H]
        \centering
        \includegraphics[width=1\textwidth]{anexo/exp/All Libraries Time Comparison/bar_plots/All Libraries Time Comparison_d_08_1_1_2_linear_bar.png}
        \caption[]{Gráfico de tiempo de ejecución de las diferentes bibliotecas al crear un gráfico para el input \textbf{d\_08\_1\_1\_2}.}
        \label{fig:all_libraries_time_comparison_plot_bar_1}
    \end{figure}
}

\DeclareRobustCommand{\AllLibrariesTimeComparisonTwoPlotLine}{
    %insertar imagen
    \begin{figure}[H]
        \centering
        \includegraphics[width=1\textwidth]{anexo/exp/All Libraries Time Comparison/plots/All Libraries Time Comparison_d_08_1_1_10_linear_line.png}
        \caption[]{Gráfico de tiempo de ejecución de las diferentes bibliotecas al crear un gráfico para el input \textbf{d\_08\_1\_1\_10}.}
        \label{fig:all_libraries_time_comparison_plot_line_2}
    \end{figure}
}

\DeclareRobustCommand{\AllLibrariesTimeComparisonTwoPlotBar}{
    %insertar imagen
    \begin{figure}[H]
        \centering
        \includegraphics[width=1\textwidth]{anexo/exp/All Libraries Time Comparison/bar_plots/All Libraries Time Comparison_d_08_1_1_10_linear_bar.png}
        \caption[]{Gráfico de tiempo de ejecución de las diferentes bibliotecas al crear un gráfico para el input \textbf{d\_08\_1\_1\_10}.}
        \label{fig:all_libraries_time_comparison_plot_bar_2}
    \end{figure}
}

\DeclareRobustCommand{\AllLibrariesTimeComparisonThreePlotLine}{
    %insertar imagen
    \begin{figure}[H]
        \centering
        \includegraphics[width=1\textwidth]{anexo/exp/All Libraries Time Comparison/plots/All Libraries Time Comparison_yellow_tripdata_2015-01_linear_line.png}
        \caption[]{Gráfico de tiempo de ejecución de las diferentes bibliotecas al crear un gráfico para el input \textbf{yellow\_tripdata\_2015\_01}.}
        \label{fig:all_libraries_time_comparison_plot_line_3}
    \end{figure}
}

\DeclareRobustCommand{\AllLibrariesTimeComparisonThreePlotBar}{
    %insertar imagen
    \begin{figure}[H]
        \centering
        \includegraphics[width=1\textwidth]{anexo/exp/All Libraries Time Comparison/bar_plots/All Libraries Time Comparison_yellow_tripdata_2015-01_linear_bar.png}
        \caption[]{Gráfico de tiempo de ejecución de las diferentes bibliotecas al crear un gráfico para el input \textbf{yellow\_tripdata\_2015\_01}.}
        \label{fig:all_libraries_time_comparison_plot_bar_3}
    \end{figure}
}

Building Time Comparison
\DeclareRobustCommand{\BuildingTimeComparisonOnePlotLine}{
    %insertar imagen
    \begin{figure}[H]
        \centering
        \includegraphics[width=1\textwidth]{anexo/exp/Building Time Comparison/plots/Building Time Comparison_d_08_1_1_2_linear_line.png}
        \caption[]{Gráfico de tiempo de construcción de las diferentes bibliotecas para el input \textbf{d\_08\_1\_1\_2}.}
        \label{fig:building_time_comparison_plot_line_1}
    \end{figure}
}

\DeclareRobustCommand{\BuildingTimeComparisonOnePlotBar}{
    %insertar imagen
    \begin{figure}[H]
        \centering
        \includegraphics[width=1\textwidth]{anexo/exp/Building Time Comparison/bar_plots/Building Time Comparison_d_08_1_1_2_linear_bar.png}
        \caption[]{Gráfico de tiempo de construcción de las diferentes bibliotecas para el input \textbf{d\_08\_1\_1\_2}.}
        \label{fig:building_time_comparison_plot_bar_1}
    \end{figure}
}

\DeclareRobustCommand{\BuildingTimeComparisonTwoPlotLine}{
    %insertar imagen
    \begin{figure}[H]
        \centering
        \includegraphics[width=1\textwidth]{anexo/exp/Building Time Comparison/plots/Building Time Comparison_d_08_1_1_10_linear_line.png}
        \caption[]{Gráfico de tiempo de construcción de las diferentes bibliotecas para el input \textbf{d\_08\_1\_1\_10}.}
        \label{fig:building_time_comparison_plot_line_2}
    \end{figure}
}

\DeclareRobustCommand{\BuildingTimeComparisonTwoPlotBar}{
    %insertar imagen
    \begin{figure}[H]
        \centering
        \includegraphics[width=1\textwidth]{anexo/exp/Building Time Comparison/bar_plots/Building Time Comparison_d_08_1_1_10_linear_bar.png}
        \caption[]{Gráfico de tiempo de construcción de las diferentes bibliotecas para el input \textbf{d\_08\_1\_1\_10}.}
        \label{fig:building_time_comparison_plot_bar_2}
    \end{figure}
}

\DeclareRobustCommand{\BuildingTimeComparisonThreePlotLine}{
    %insertar imagen
    \begin{figure}[H]
        \centering
        \includegraphics[width=1\textwidth]{anexo/exp/Building Time Comparison/plots/Building Time Comparison_yellow_tripdata_2015-01_linear_line.png}
        \caption[]{Gráfico de tiempo de construcción de las diferentes bibliotecas para el input \textbf{yellow\_tripdata\_2015\_01}.}
        \label{fig:building_time_comparison_plot_line_3}
    \end{figure}
}

\DeclareRobustCommand{\BuildingTimeComparisonThreePlotBar}{
    %insertar imagen
    \begin{figure}[H]
        \centering
        \includegraphics[width=1\textwidth]{anexo/exp/Building Time Comparison/bar_plots/Building Time Comparison_yellow_tripdata_2015-01_linear_bar.png}
        \caption[]{Gráfico de tiempo de construcción de las diferentes bibliotecas para el input \textbf{yellow\_tripdata\_2015\_01}.}
        \label{fig:building_time_comparison_plot_bar_3}
    \end{figure}
}




% Comparison of Space Used
\DeclareRobustCommand{\ComparisonOfSpaceUsedOnePlotLine}{
    %insertar imagen
    \begin{figure}[H]
        \centering
        \includegraphics[width=1\textwidth]{anexo/exp/Comparison of Space Used/plots/Comparison of Space Used_d_08_1_1_2_linear_line.png}
        \caption[]{Gráfico de espacio usado por las diferentes bibliotecas para el input \textbf{d\_08\_1\_1\_2}.}
        \label{fig:comparison_of_space_used_plot_line_1}
    \end{figure}
}

\DeclareRobustCommand{\ComparisonOfSpaceUsedOnePlotBar}{
    %insertar imagen
    \begin{figure}[H]
        \centering
        \includegraphics[width=1\textwidth]{anexo/exp/Comparison of Space Used/bar_plots/Comparison of Space Used_d_08_1_1_2_log_bar.png}
        \caption[]{Gráfico de espacio usado por las diferentes bibliotecas para el input \textbf{d\_08\_1\_1\_2}.}
        \label{fig:comparison_of_space_used_plot_bar_1}
    \end{figure}
    \begin{table}[H]
        \centering
        \resizebox{\textwidth}{!}{%
        \begin{tabular}{|l|c|c|c|c|c|c|}
        \hline\multicolumn{1}{|c|}{Option} & \multicolumn{6}{c|}{\textbf{Number of data points}} \\
        \cline{2-7}
         & \textbf{6000} & \textbf{12000} & \textbf{20000} & \textbf{70000} & \textbf{100000} & \textbf{300000} \\
        \hline
        Compressed Vector Downsample - dac\_vector & 1.66e+03 [B] & 2.68e+03 [B] & 4.07e+03 [B] & 9.49e+03 [B] & 9.48e+03 [B] & 9.54e+03 [B] \\
        Compressed Vector Downsample - vlc\_vector\_fibonacci & 1.24e+03 [B] & 2.26e+03 [B] & 3.65e+03 [B] & 8.89e+03 [B] & 8.88e+03 [B] & 9.04e+03 [B] \\
        Original Data & 1.06e+05 [B] & 2.16e+05 [B] & 3.46e+05 [B] & 1.12e+06 [B] & 1.60e+06 [B] & 5.20e+06 [B] \\
        TS Downsample & 2.14e+03 [B] & 4.06e+03 [B] & 6.62e+03 [B] & 1.62e+04 [B] & 1.62e+04 [B] & 1.62e+04 [B] \\
        \hline
        \end{tabular}
        }
        \label{tab:comparison of space used-d-08-1-1-2}
    \end{table}
    \begin{table}[H]
        \centering
        \caption{Comparison of Space Used\_d\_08\_1\_1\_2 – Tasa de compactación CVD vs TS vs Datos Originales}
        \label{tab:Comparison of Space Used\_d\_08\_1\_1\_2_cvd_compactacion}
        \begin{tabular}{rlrr}
        \toprule
        n\_size & Variante & Comp. vs TS (\%) & Comp. vs Original (\%) \\
        \midrule
        6000 & CVD - dac\_vector & 22.48 & 98.43 \\
        6000 & CVD - vlc\_vector\_fibonacci & 42.26 & 98.83 \\
        12000 & CVD - dac\_vector & 34.10 & 98.76 \\
        12000 & CVD - vlc\_vector\_fibonacci & 44.34 & 98.95 \\
        20000 & CVD - dac\_vector & 38.56 & 98.82 \\
        20000 & CVD - vlc\_vector\_fibonacci & 44.96 & 98.95 \\
        70000 & CVD - dac\_vector & 41.53 & 99.16 \\
        70000 & CVD - vlc\_vector\_fibonacci & 45.18 & 99.21 \\
        100000 & CVD - dac\_vector & 41.58 & 99.41 \\
        100000 & CVD - vlc\_vector\_fibonacci & 45.28 & 99.45 \\
        300000 & CVD - dac\_vector & 41.19 & 99.82 \\
        300000 & CVD - vlc\_vector\_fibonacci & 44.29 & 99.83 \\
        \bottomrule
        \end{tabular}
        
    \end{table}
}

\DeclareRobustCommand{\ComparisonOfSpaceUsedTwoPlotLine}{
    %insertar imagen
    \begin{figure}[H]
        \centering
        \includegraphics[width=1\textwidth]{anexo/exp/Comparison of Space Used/plots/Comparison of Space Used_d_08_1_1_10_linear_line.png}
        \caption[]{Gráfico de espacio usado por las diferentes bibliotecas para el input \textbf{d\_08\_1\_1\_10}.}
        \label{fig:comparison_of_space_used_plot_line_2}
    \end{figure}
}

\DeclareRobustCommand{\ComparisonOfSpaceUsedTwoPlotBar}{
    %insertar imagen
    \begin{figure}[H]
        \centering
        \includegraphics[width=1\textwidth]{anexo/exp/Comparison of Space Used/bar_plots/Comparison of Space Used_d_08_1_1_10_log_bar.png}
        \caption[]{Gráfico de espacio usado por las diferentes bibliotecas para el input \textbf{d\_08\_1\_1\_10}.}
        \label{fig:comparison_of_space_used_plot_bar_2}
    \end{figure}

    \begin{table}[H]
        \centering
        \resizebox{\textwidth}{!}{%
        \begin{tabular}{|l|c|c|c|c|c|c|}
        \hline\multicolumn{1}{|c|}{Option} & \multicolumn{6}{c|}{\textbf{Number of data points}} \\
        \cline{2-7}
         & \textbf{6000} & \textbf{12000} & \textbf{20000} & \textbf{70000} & \textbf{100000} & \textbf{300000} \\
        \hline
        Compressed Vector Downsample - dac\_vector & 1.67e+03 [B] & 2.73e+03 [B] & 4.21e+03 [B] & 9.57e+03 [B] & 9.55e+03 [B] & 9.57e+03 [B] \\
        Compressed Vector Downsample - vlc\_vector\_fibonacci & 1.28e+03 [B] & 2.33e+03 [B] & 3.77e+03 [B] & 9.15e+03 [B] & 9.13e+03 [B] & 9.17e+03 [B] \\
        Original Data & 1.06e+05 [B] & 2.16e+05 [B] & 3.46e+05 [B] & 1.12e+06 [B] & 1.60e+06 [B] & 5.20e+06 [B] \\
        TS Downsample & 2.18e+03 [B] & 4.10e+03 [B] & 6.66e+03 [B] & 1.62e+04 [B] & 1.62e+04 [B] & 1.62e+04 [B] \\
        \hline
        \end{tabular}
        }
        \label{tab:comparison of space used-d-08-1-1-10}
    \end{table}
        \begin{table}[H]
\centering
\caption{Comparison of Space Used\_d\_08\_1\_1\_10 – Tasa de compactación CVD vs TS vs Datos Originales}
\label{tab:Comparison of Space Used\_d\_08\_1\_1\_10_cvd_compactacion}
\begin{tabular}{rlrr}
\toprule
n\_size & Variante & Comp. vs TS (\%) & Comp. vs Original (\%) \\
\midrule
6000 & CVD - dac\_vector & 23.25 & 98.43 \\
6000 & CVD - vlc\_vector\_fibonacci & 41.27 & 98.80 \\
12000 & CVD - dac\_vector & 33.45 & 98.74 \\
12000 & CVD - vlc\_vector\_fibonacci & 43.02 & 98.92 \\
20000 & CVD - dac\_vector & 36.81 & 98.78 \\
20000 & CVD - vlc\_vector\_fibonacci & 43.42 & 98.91 \\
70000 & CVD - dac\_vector & 41.04 & 99.15 \\
70000 & CVD - vlc\_vector\_fibonacci & 43.60 & 99.19 \\
100000 & CVD - dac\_vector & 41.14 & 99.40 \\
100000 & CVD - vlc\_vector\_fibonacci & 43.75 & 99.43 \\
300000 & CVD - dac\_vector & 41.04 & 99.82 \\
300000 & CVD - vlc\_vector\_fibonacci & 43.45 & 99.82 \\
\bottomrule
\end{tabular}

\end{table}
}

\DeclareRobustCommand{\ComparisonOfSpaceUsedThreePlotLine}{
    %insertar imagen
    \begin{figure}[H]
        \centering
        \includegraphics[width=1\textwidth]{anexo/exp/Comparison of Space Used/plots/Comparison of Space Used_yellow_tripdata_2015-01_linear_line.png}
        \caption[]{Gráfico de espacio usado por las diferentes bibliotecas para el input \textbf{yellow\_tripdata\_2015\_01}.}
        \label{fig:comparison_of_space_used_plot_line_3}
    \end{figure}
}

\DeclareRobustCommand{\ComparisonOfSpaceUsedThreePlotBar}{
    %insertar imagen
    \begin{figure}[H]
        \centering
        \includegraphics[width=1\textwidth]{anexo/exp/Comparison of Space Used/bar_plots/Comparison of Space Used_yellow_tripdata_2015-01_log_bar.png}
        \caption[]{Gráfico de espacio usado por las diferentes bibliotecas para el input \textbf{yellow\_tripdata\_2015\_01}.}
        \label{fig:comparison_of_space_used_plot_bar_3}
    \end{figure}
    \begin{table}[H]
        \centering
        \resizebox{\textwidth}{!}{%
        \begin{tabular}{|l|c|c|c|c|c|c|}
        \hline\multicolumn{1}{|c|}{Option} & \multicolumn{6}{c|}{\textbf{Number of data points}} \\
        \cline{2-7}
         & \textbf{6000} & \textbf{12000} & \textbf{20000} & \textbf{70000} & \textbf{100000} & \textbf{300000} \\
        \hline
        Compressed Vector Downsample - dac\_vector & 4.68e+02 [B] & 4.60e+02 [B] & 4.60e+02 [B] & 4.60e+02 [B] & 4.76e+02 [B] & 4.76e+02 [B] \\
        Compressed Vector Downsample - vlc\_vector\_fibonacci & 1.80e+02 [B] & 1.72e+02 [B] & 1.72e+02 [B] & 1.72e+02 [B] & 1.96e+02 [B] & 1.96e+02 [B] \\
        Original Data & 1.06e+05 [B] & 2.16e+05 [B] & 3.46e+05 [B] & 1.12e+06 [B] & 1.60e+06 [B] & 5.20e+06 [B] \\
        TS Downsample & 2.88e+02 [B] & 2.56e+02 [B] & 2.56e+02 [B] & 2.56e+02 [B] & 3.20e+02 [B] & 3.20e+02 [B] \\
        \hline
        \end{tabular}
        }
        \label{tab:comparison of space used-yellow-tripdata-2015-01}
    \end{table}
    \begin{table}[H]
\centering
\caption{Comparison of Space Used\_yellow\_tripdata\_2015-01 – Tasa de compactación CVD vs TS vs Datos Originales}
\label{tab:Comparison of Space Used\_yellow\_tripdata\_2015-01_cvd_compactacion}
\begin{tabular}{rlrr}
\toprule
n\_size & Variante & Comp. vs TS (\%) & Comp. vs Original (\%) \\
\midrule
6000 & CVD - dac\_vector & -62.50 & 99.56 \\
6000 & CVD - vlc\_vector\_fibonacci & 37.50 & 99.83 \\
12000 & CVD - dac\_vector & -79.69 & 99.79 \\
12000 & CVD - vlc\_vector\_fibonacci & 32.81 & 99.92 \\
20000 & CVD - dac\_vector & -79.69 & 99.87 \\
20000 & CVD - vlc\_vector\_fibonacci & 32.81 & 99.95 \\
70000 & CVD - dac\_vector & -79.69 & 99.96 \\
70000 & CVD - vlc\_vector\_fibonacci & 32.81 & 99.98 \\
100000 & CVD - dac\_vector & -48.75 & 99.97 \\
100000 & CVD - vlc\_vector\_fibonacci & 38.75 & 99.99 \\
300000 & CVD - dac\_vector & -48.75 & 99.99 \\
300000 & CVD - vlc\_vector\_fibonacci & 38.75 & 100.00 \\
\bottomrule
\end{tabular}

\end{table}
}



% ===========================
% CVD Decimal Places Access Time Comparison
% ===========================
\DeclareRobustCommand{\CVDDecimalPlacesAccessTimeComparisonOnePlotLine}{
    \begin{figure}[H]
        \centering
        \includegraphics[width=1\textwidth]{anexo/exp/CVD Decimal Places Access Time Comparison/plots/CVD Decimal Places Access Time Comparison_d_08_1_1_2_linear_line.png}
        \caption[]{Gráfico de tiempo de acceso CVD con diferentes lugares decimales para el input \textbf{d\_08\_1\_1\_2}.}
        \label{fig:cvd_decimal_places_access_time_comparison_plot_line_1}
    \end{figure}
    \begin{figure}[H]
        \centering
        \includegraphics[width=1\textwidth]{anexo/exp/CVD Decimal Places Access Time Comparison/plots/CVD Decimal Places Access Time Comparison_d_08_1_1_2_log_line.png}
        \caption[]{Gráfico de tiempo de acceso CVD con diferentes lugares decimales para el input \textbf{d\_08\_1\_1\_2} en escala logarítmica.}
        \label{fig:cvd_decimal_places_access_time_comparison_plot_log_1}
    \end{figure}
}

\DeclareRobustCommand{\CVDDecimalPlacesAccessTimeComparisonTwoPlotLine}{
    \begin{figure}[H]
        \centering
        \includegraphics[width=1\textwidth]{anexo/exp/CVD Decimal Places Access Time Comparison/plots/CVD Decimal Places Access Time Comparison_d_08_1_1_10_linear_line.png}
        \caption[]{Gráfico de tiempo de acceso CVD con diferentes lugares decimales para el input \textbf{d\_08\_1\_1\_10}.}
        \label{fig:cvd_decimal_places_access_time_comparison_plot_line_2}
    \end{figure}
    \begin{figure}[H]
        \centering
        \includegraphics[width=1\textwidth]{anexo/exp/CVD Decimal Places Access Time Comparison/plots/CVD Decimal Places Access Time Comparison_d_08_1_1_10_log_line.png}
        \caption[]{Gráfico de tiempo de acceso CVD con diferentes lugares decimales para el input \textbf{d\_08\_1\_1\_10} en escala logarítmica.}
        \label{fig:cvd_decimal_places_access_time_comparison_plot_log_2}
    \end{figure}
}

\DeclareRobustCommand{\CVDDecimalPlacesAccessTimeComparisonThreePlotLine}{
    \begin{figure}[H]
        \centering
        \includegraphics[width=1\textwidth]{anexo/exp/CVD Decimal Places Access Time Comparison/plots/CVD Decimal Places Access Time Comparison_yellow_tripdata_2015-01_linear_line.png}
        \caption[]{Gráfico de tiempo de acceso CVD con diferentes lugares decimales para el input \textbf{yellow\_tripdata\_2015\_01}.}
        \label{fig:cvd_decimal_places_access_time_comparison_plot_line_3}
    \end{figure}
    \begin{figure}[H]
        \centering
        \includegraphics[width=1\textwidth]{anexo/exp/CVD Decimal Places Access Time Comparison/plots/CVD Decimal Places Access Time Comparison_yellow_tripdata_2015-01_log_line.png}
        \caption[]{Gráfico de tiempo de acceso CVD con diferentes lugares decimales para el input \textbf{yellow\_tripdata\_2015\_01} en escala logarítmica.}
        \label{fig:cvd_decimal_places_access_time_comparison_plot_log_3}
    \end{figure}
}


% ===========================
% CVD Decimal Places Build Time Comparison
% ===========================
\DeclareRobustCommand{\CVDDecimalPlacesBuildTimeComparisonOnePlotLine}{
    \begin{figure}[H]
        \centering
        \includegraphics[width=1\textwidth]{anexo/exp/CVD Decimal Places Build Time Comparison/plots/CVD Decimal Places Build Time Comparison_d_08_1_1_2_linear_line.png}
        \caption[]{Gráfico de tiempo de construcción CVD con diferentes lugares decimales para el input \textbf{d\_08\_1\_1\_2}.}
        \label{fig:cvd_decimal_places_build_time_comparison_plot_line_1}
    \end{figure}
    \begin{figure}[H]
        \centering
        \includegraphics[width=1\textwidth]{anexo/exp/CVD Decimal Places Build Time Comparison/plots/CVD Decimal Places Build Time Comparison_d_08_1_1_2_log_line.png}
        \caption[]{Gráfico de tiempo de construcción CVD con diferentes lugares decimales para el input \textbf{d\_08\_1\_1\_2} en escala logarítmica.}
        \label{fig:cvd_decimal_places_build_time_comparison_plot_log_1}
    \end{figure}
}

\DeclareRobustCommand{\CVDDecimalPlacesBuildTimeComparisonTwoPlotLine}{
    \begin{figure}[H]
        \centering
        \includegraphics[width=1\textwidth]{anexo/exp/CVD Decimal Places Build Time Comparison/plots/CVD Decimal Places Build Time Comparison_d_08_1_1_10_linear_line.png}
        \caption[]{Gráfico de tiempo de construcción CVD con diferentes lugares decimales para el input \textbf{d\_08\_1\_1\_10}.}
        \label{fig:cvd_decimal_places_build_time_comparison_plot_line_2}
    \end{figure}
    \begin{figure}[H]
        \centering
        \includegraphics[width=1\textwidth]{anexo/exp/CVD Decimal Places Build Time Comparison/plots/CVD Decimal Places Build Time Comparison_d_08_1_1_10_log_line.png}
        \caption[]{Gráfico de tiempo de construcción CVD con diferentes lugares decimales para el input \textbf{d\_08\_1\_1\_10} en escala logarítmica.}
        \label{fig:cvd_decimal_places_build_time_comparison_plot_log_2}
    \end{figure}
}

\DeclareRobustCommand{\CVDDecimalPlacesBuildTimeComparisonThreePlotLine}{
    \begin{figure}[H]
        \centering
        \includegraphics[width=1\textwidth]{anexo/exp/CVD Decimal Places Build Time Comparison/plots/CVD Decimal Places Build Time Comparison_yellow_tripdata_2015-01_linear_line.png}
        \caption[]{Gráfico de tiempo de construcción CVD con diferentes lugares decimales para el input \textbf{yellow\_tripdata\_2015\_01}.}
        \label{fig:cvd_decimal_places_build_time_comparison_plot_line_3}
    \end{figure}
    \begin{figure}[H]
        \centering
        \includegraphics[width=1\textwidth]{anexo/exp/CVD Decimal Places Build Time Comparison/plots/CVD Decimal Places Build Time Comparison_yellow_tripdata_2015-01_log_line.png}
        \caption[]{Gráfico de tiempo de construcción CVD con diferentes lugares decimales para el input \textbf{yellow\_tripdata\_2015\_01} en escala logarítmica.}
        \label{fig:cvd_decimal_places_build_time_comparison_plot_log_3}
    \end{figure}
}


% ===========================
% CVD Decimal Places Size Comparison
% ===========================
\DeclareRobustCommand{\CVDDecimalPlacesSizeComparisonOnePlotLine}{
    \begin{figure}[H]
        \centering
        \includegraphics[width=1\textwidth]{anexo/exp/CVD Decimal Places Size Comparison/plots/CVD Decimal Places Size Comparison_d_08_1_1_2_linear_line.png}
        \caption[]{Gráfico de tamaño CVD con diferentes lugares decimales para el input \textbf{d\_08\_1\_1\_2}.}
        \label{fig:cvd_decimal_places_size_comparison_plot_line_1}
    \end{figure}
    \begin{figure}[H]
        \centering
        \includegraphics[width=1\textwidth]{anexo/exp/CVD Decimal Places Size Comparison/plots/CVD Decimal Places Size Comparison_d_08_1_1_2_log_line.png}
        \caption[]{Gráfico de tamaño CVD con diferentes lugares decimales para el input \textbf{d\_08\_1\_1\_2} en escala logarítmica.}
        \label{fig:cvd_decimal_places_size_comparison_plot_log_1}
    \end{figure}
}

\DeclareRobustCommand{\CVDDecimalPlacesSizeComparisonTwoPlotLine}{
    \begin{figure}[H]
        \centering
        \includegraphics[width=1\textwidth]{anexo/exp/CVD Decimal Places Size Comparison/plots/CVD Decimal Places Size Comparison_d_08_1_1_10_linear_line.png}
        \caption[]{Gráfico de tamaño CVD con diferentes lugares decimales para el input \textbf{d\_08\_1\_1\_10}.}
        \label{fig:cvd_decimal_places_size_comparison_plot_line_2}
    \end{figure}
    \begin{figure}[H]
        \centering
        \includegraphics[width=1\textwidth]{anexo/exp/CVD Decimal Places Size Comparison/plots/CVD Decimal Places Size Comparison_d_08_1_1_10_log_line.png}
        \caption[]{Gráfico de tamaño CVD con diferentes lugares decimales para el input \textbf{d\_08\_1\_1\_10} en escala logarítmica.}
        \label{fig:cvd_decimal_places_size_comparison_plot_log_2}
    \end{figure}
}

\DeclareRobustCommand{\CVDDecimalPlacesSizeComparisonThreePlotLine}{
    \begin{figure}[H]
        \centering
        \includegraphics[width=1\textwidth]{anexo/exp/CVD Decimal Places Size Comparison/plots/CVD Decimal Places Size Comparison_yellow_tripdata_2015-01_linear_line.png}
        \caption[]{Gráfico de tamaño CVD con diferentes lugares decimales para el input \textbf{yellow\_tripdata\_2015\_01}.}
        \label{fig:cvd_decimal_places_size_comparison_plot_line_3}
    \end{figure}
    \begin{figure}[H]
        \centering
        \includegraphics[width=1\textwidth]{anexo/exp/CVD Decimal Places Size Comparison/plots/CVD Decimal Places Size Comparison_yellow_tripdata_2015-01_log_line.png}
        \caption[]{Gráfico de tamaño CVD con diferentes lugares decimales para el input \textbf{yellow\_tripdata\_2015\_01} en escala logarítmica.}
        \label{fig:cvd_decimal_places_size_comparison_plot_log_3}
    \end{figure}
}





% PyGal Plotting Memory Allocation
\DeclareRobustCommand{\PyGalMemoryAllocationOnePlotLine}{
    %insertar imagen
    \begin{figure}[H]
        \centering
        \includegraphics[width=1\textwidth]{anexo/exp/Pygal Plotting Memory Allocation/plots/Pygal Plotting Memory Allocation_d_08_1_1_2_log_line.png}
        \caption[]{Gráfico de memoria asignada por PyGal para el input \textbf{d\_08\_1\_1\_2}.}
        \label{fig:pygal_memory_allocation_plot_line_1}
    \end{figure}
}

\DeclareRobustCommand{\PyGalMemoryAllocationOnePlotBar}{
    %insertar imagen
    \begin{figure}[H]
        \centering
        \includegraphics[width=1\textwidth]{anexo/exp/Pygal Plotting Memory Allocation/bar_plots/Pygal Plotting Memory Allocation_d_08_1_1_2_log_bar.png}
        \caption[]{Gráfico de memoria asignada por PyGal para el input \textbf{d\_08\_1\_1\_2}.}
        \label{fig:pygal_memory_allocation_plot_bar_1}
    \end{figure}
\begin{table}[H]
\centering
\resizebox{\textwidth}{!}{%
\begin{tabular}{|l|c|c|c|c|c|c|}
\hline\multicolumn{1}{|c|}{Option} & \multicolumn{6}{c|}{\textbf{Number of data points}} \\
\cline{2-7}
 & \textbf{6000} & \textbf{12000} & \textbf{20000} & \textbf{70000} & \textbf{100000} & \textbf{300000} \\
\hline
Compressed Vector Downsample - dac\_vector & 9.38e+02 [kb] & 1.75e+03 [kb] & 2.81e+03 [kb] & 6.88e+03 [kb] & 6.88e+03 [kb] & 6.88e+03 [kb] \\
Compressed Vector Downsample - vlc\_vector\_fibonacci & 9.38e+02 [kb] & 1.75e+03 [kb] & 2.81e+03 [kb] & 6.88e+03 [kb] & 6.88e+03 [kb] & 6.88e+03 [kb] \\
Original Data & 4.02e+04 [kb] & 8.03e+04 [kb] & 1.34e+05 [kb] & 4.70e+05 [kb] & 6.71e+05 [kb] & 2.01e+06 [kb] \\
TS Downsample & 9.37e+02 [kb] & 1.75e+03 [kb] & 2.81e+03 [kb] & 6.88e+03 [kb] & 6.88e+03 [kb] & 6.88e+03 [kb] \\
\hline
\end{tabular}
}
\label{tab:pygal plotting memory allocation-d-08-1-1-2}
\end{table}
   \begin{table}[H]
\centering
\caption{\textit{d\_08\_1\_1\_2} – Memoria por elemento}
\label{tab:d\_08\_1\_1\_2_mem_por_elemento_sin_original}
\begin{tabular}{rrrr}
\toprule
 & CVD - dac\_vector & CVD - vlc\_vector\_fibonacci & TS Downsample \\
n\_size &  &  &  \\
\midrule
6000 & 0.94 & 0.94 & 0.94 \\
12000 & 1.75 & 1.75 & 1.75 \\
20000 & 2.81 & 2.81 & 2.81 \\
70000 & 6.88 & 6.88 & 6.88 \\
100000 & 6.88 & 6.88 & 6.88 \\
300000 & 6.88 & 6.88 & 6.88 \\
\bottomrule
\end{tabular}

\end{table}
}

\DeclareRobustCommand{\PyGalMemoryAllocationTwoPlotLine}{
    %insertar imagen
    \begin{figure}[H]
        \centering
        \includegraphics[width=1\textwidth]{anexo/exp/PyGal Plotting Memory Allocation/plots/PyGal Plotting Memory Allocation_d_08_1_1_10_linear_line.png}
        \caption[]{Gráfico de memoria asignada por PyGal para el input \textbf{d\_08\_1\_1\_10}.}
        \label{fig:pygal_memory_allocation_plot_line_2}
    \end{figure}
}

\DeclareRobustCommand{\PyGalMemoryAllocationTwoPlotBar}{
    %insertar imagen
    \begin{figure}[H]
        \centering
        \includegraphics[width=1\textwidth]{anexo/exp/Pygal Plotting Memory Allocation/bar_plots/Pygal Plotting Memory Allocation_d_08_1_1_10_log_bar.png}
        \caption[]{Gráfico de memoria asignada por PyGal para el input \textbf{d\_08\_1\_1\_10}.}
        \label{fig:pygal_memory_allocation_plot_bar_2}
    \end{figure}

\begin{table}[H]
\centering
\resizebox{\textwidth}{!}{%
\begin{tabular}{|l|c|c|c|c|c|c|}
\hline\multicolumn{1}{|c|}{Option} & \multicolumn{6}{c|}{\textbf{Number of data points}} \\
\cline{2-7}
 & \textbf{6000} & \textbf{12000} & \textbf{20000} & \textbf{70000} & \textbf{100000} & \textbf{300000} \\
\hline
Compressed Vector Downsample - dac_vector & 9.36e+02 [kb] & 1.74e+03 [kb] & 2.83e+03 [kb] & 6.88e+03 [kb] & 6.88e+03 [kb] & 6.89e+03 [kb] \\
Compressed Vector Downsample - vlc_vector_fibonacci & 9.36e+02 [kb] & 1.75e+03 [kb] & 2.83e+03 [kb] & 6.88e+03 [kb] & 6.88e+03 [kb] & 6.89e+03 [kb] \\
Original Data & 4.01e+04 [kb] & 8.01e+04 [kb] & 1.33e+05 [kb] & 4.70e+05 [kb] & 6.71e+05 [kb] & 2.02e+06 [kb] \\
TS Downsample & 9.36e+02 [kb] & 1.74e+03 [kb] & 2.83e+03 [kb] & 6.88e+03 [kb] & 6.88e+03 [kb] & 6.90e+03 [kb] \\
\hline
\end{tabular}
}
\label{tab:pygal plotting memory allocation-d-08-1-1-10}
\end{table}
\begin{table}[H]
\centering
\caption{\textit{d\_08\_1\_1\_10} – Memoria por elemento}
\label{tab:d\_08\_1\_1\_10_mem_por_elemento_sin_original}
\begin{tabular}{rrrr}
\toprule
 & CVD - dac\_vector & CVD - vlc\_vector\_fibonacci & TS Downsample \\
n\_size &  &  &  \\
\midrule
6000 & 0.94 & 0.94 & 0.94 \\
12000 & 1.74 & 1.75 & 1.74 \\
20000 & 2.83 & 2.83 & 2.83 \\
70000 & 6.88 & 6.88 & 6.88 \\
100000 & 6.88 & 6.88 & 6.88 \\
300000 & 6.89 & 6.89 & 6.90 \\
\bottomrule
\end{tabular}

\end{table}
}

\DeclareRobustCommand{\PyGalMemoryAllocationThreePlotLine}{
    %insertar imagen
    \begin{figure}[H]
        \centering
        \includegraphics[width=1\textwidth]{anexo/exp/PyGal Plotting Memory Allocation/plots/PyGal Plotting Memory Allocation_yellow_tripdata_2015-01_linear_line.png}
        \caption[]{Gráfico de memoria asignada por PyGal para el input \textbf{yellow\_tripdata\_2015\_01}.}
        \label{fig:pygal_memory_allocation_plot_line_3}
    \end{figure}
}

\DeclareRobustCommand{\PyGalMemoryAllocationThreePlotBar}{
    %insertar imagen
    \begin{figure}[H]
        \centering
        \includegraphics[width=1\textwidth]{anexo/exp/Pygal Plotting Memory Allocation/bar_plots/Pygal Plotting Memory Allocation_yellow_tripdata_2015-01_log_bar.png}
        \caption[]{Gráfico de memoria asignada por PyGal para el input \textbf{yellow\_tripdata\_2015\_01}.}
        \label{fig:pygal_memory_allocation_plot_bar_3}
    \end{figure}
\begin{table}[H]
\centering
\resizebox{\textwidth}{!}{%
\begin{tabular}{|l|c|c|c|c|c|c|}
\hline\multicolumn{1}{|c|}{Option} & \multicolumn{6}{c|}{\textbf{Number of data points}} \\
\cline{2-7}
 & \textbf{6000} & \textbf{12000} & \textbf{20000} & \textbf{70000} & \textbf{100000} & \textbf{300000} \\
\hline
Compressed Vector Downsample - dac_vector & 1.33e+02 [kb] & 1.09e+02 [kb] & 1.09e+02 [kb] & 1.09e+02 [kb] & 1.53e+02 [kb] & 1.52e+02 [kb] \\
Compressed Vector Downsample - vlc_vector_fibonacci & 1.33e+02 [kb] & 1.09e+02 [kb] & 1.10e+02 [kb] & 1.09e+02 [kb] & 1.53e+02 [kb] & 1.53e+02 [kb] \\
Original Data & 4.00e+04 [kb] & 7.99e+04 [kb] & 1.33e+05 [kb] & 4.65e+05 [kb] & 6.64e+05 [kb] & 2.00e+06 [kb] \\
TS Downsample & 1.32e+02 [kb] & 1.09e+02 [kb] & 1.09e+02 [kb] & 1.08e+02 [kb] & 1.53e+02 [kb] & 1.53e+02 [kb] \\
\hline
\end{tabular}
}
\label{tab:pygal plotting memory allocation-yellow-tripdata-2015-01}
\end{table}
\begin{table}[H]
\centering
\caption{\textit{yellow\_tripdata\_2015-01} – Memoria por elemento}
\label{tab:yellow\_tripdata\_2015-01_mem_por_elemento_sin_original}
\begin{tabular}{rrrr}
\toprule
 & CVD - dac\_vector & CVD - vlc\_vector\_fibonacci & TS Downsample \\
n\_size &  &  &  \\
\midrule
6000 & 0.13 & 0.13 & 0.13 \\
12000 & 0.11 & 0.11 & 0.11 \\
20000 & 0.11 & 0.11 & 0.11 \\
70000 & 0.11 & 0.11 & 0.11 \\
100000 & 0.15 & 0.15 & 0.15 \\
300000 & 0.15 & 0.15 & 0.15 \\
\bottomrule
\end{tabular}

\end{table}
}

% PyGal Plot Time
\DeclareRobustCommand{\PyGalPlotTimeOnePlotLine}{
    %insertar imagen
    \begin{figure}[H]
        \centering
        \includegraphics[width=1\textwidth]{anexo/exp/PyGal Plot Time/plots/PyGal Plot Time_d_08_1_1_2_linear_line.png}
        \caption[]{Gráfico de tiempo de renderización PyGal para el input \textbf{d\_08\_1\_1\_2}.}
        \label{fig:pygal_plot_time_plot_line_1}
    \end{figure}
    \begin{figure}[H]
        \centering
        \includegraphics[width=1\textwidth]{anexo/exp/PyGal Plot Time/plots/PyGal Plot Time_d_08_1_1_2_log_line.png}
        \caption[]{Gráfico de tiempo de renderización PyGal para el input \textbf{d\_08\_1\_1\_2} en escala logarítmica.}
        \label{fig:pygal_plot_time_plot_line_1_log}
    \end{figure}
}

\DeclareRobustCommand{\PyGalPlotTimeOnePlotBar}{
    %insertar imagen
    \begin{figure}[H]
        \centering
        \includegraphics[width=1\textwidth]{anexo/exp/PyGal Plot Time/bar_plots/PyGal Plot Time_d_08_1_1_2_linear_bar.png}
        \caption[]{Gráfico de tiempo de renderización PyGal para el input \textbf{d\_08\_1\_1\_2}.}
        \label{fig:pygal_plot_time_plot_bar_1}
    \end{figure}
}

\DeclareRobustCommand{\PyGalPlotTimeTwoPlotLine}{
    %insertar imagen
    \begin{figure}[H]
        \centering
        \includegraphics[width=1\textwidth]{anexo/exp/PyGal Plot Time/plots/PyGal Plot Time_d_08_1_1_10_linear_line.png}
        \caption[]{Gráfico de tiempo de renderización PyGal para el input \textbf{d\_08\_1\_1\_10}.}
        \label{fig:pygal_plot_time_plot_line_2}
    \end{figure}
    \begin{figure}[H]
        \centering
        \includegraphics[width=1\textwidth]{anexo/exp/PyGal Plot Time/plots/PyGal Plot Time_d_08_1_1_10_log_line.png}
        \caption[]{Gráfico de tiempo de renderización PyGal para el input \textbf{d\_08\_1\_1\_10} en escala logarítmica.}
        \label{fig:pygal_plot_time_plot_line_2_log}
    \end{figure}
}

\DeclareRobustCommand{\PyGalPlotTimeTwoPlotBar}{
    %insertar imagen
    \begin{figure}[H]
        \centering
        \includegraphics[width=1\textwidth]{anexo/exp/PyGal Plot Time/bar_plots/PyGal Plot Time_d_08_1_1_10_linear_bar.png}
        \caption[]{Gráfico de tiempo de renderización PyGal para el input \textbf{d\_08\_1\_1\_10}.}
        \label{fig:pygal_plot_time_plot_bar_2}
    \end{figure}
}

\DeclareRobustCommand{\PyGalPlotTimeThreePlotLine}{
    %insertar imagen
    \begin{figure}[H]
        \centering
        \includegraphics[width=1\textwidth]{anexo/exp/PyGal Plot Time/plots/PyGal Plot Time_yellow_tripdata_2015-01_linear_line.png}
        \caption[]{Gráfico de tiempo de renderización PyGal para el input \textbf{yellow\_tripdata\_2015\_01}.}
        \label{fig:pygal_plot_time_plot_line_3}
    \end{figure}
    \begin{figure}[H]
        \centering
        \includegraphics[width=1\textwidth]{anexo/exp/PyGal Plot Time/plots/PyGal Plot Time_yellow_tripdata_2015-01_log_line.png}
        \caption[]{Gráfico de tiempo de renderización PyGal para el input \textbf{yellow\_tripdata\_2015\_01}.}
        \label{fig:pygal_plot_time_plot_line_3_log}
    \end{figure}
}

\DeclareRobustCommand{\PyGalPlotTimeThreePlotBar}{
    %insertar imagen
    \begin{figure}[H]
        \centering
        \includegraphics[width=1\textwidth]{anexo/exp/PyGal Plot Time/bar_plots/PyGal Plot Time_yellow_tripdata_2015-01_linear_bar.png}
        \caption[]{Gráfico de tiempo de renderización PyGal para el input \textbf{yellow\_tripdata\_2015\_01}.}
        \label{fig:pygal_plot_time_plot_bar_3}
    \end{figure}
}

\DeclareRobustCommand{\SDSLFourPyAccessTimeComparisonOnePlotBar}{
    \begin{figure}[H]
        \centering
        \includegraphics[width=1\textwidth]{anexo/exp/SDSL4Py Access Time Comparison/bar_plots/SDSL4Py Access Time Comparison_d_08_1_1_2_log_bar.png}
        \caption[]{Gráfico de tiempo de acceso SDSL4Py en escala logarítmica para el input \textbf{d\_08\_1\_1\_2}.}
        \label{fig:sdsl4py_access_time_comparison_plot_log_bar_1}
    \end{figure}
}

\DeclareRobustCommand{\SDSLFourPyAccessTimeComparisonTwoPlotBar}{
    \begin{figure}[H]
        \centering
        \includegraphics[width=1\textwidth]{anexo/exp/SDSL4Py Access Time Comparison/bar_plots/SDSL4Py Access Time Comparison_d_08_1_1_10_log_bar.png}
        \caption[]{Gráfico de tiempo de acceso SDSL4Py en escala logarítmica para el input \textbf{d\_08\_1\_1\_10}.}
        \label{fig:sdsl4py_access_time_comparison_plot_log_bar_2}
    \end{figure}
}

\DeclareRobustCommand{\SDSLFourPyAccessTimeComparisonThreePlotBar}{
    \begin{figure}[H]
        \centering
        \includegraphics[width=1\textwidth]{anexo/exp/SDSL4Py Access Time Comparison/bar_plots/SDSL4Py Access Time Comparison_yellow_tripdata_2015-01_log_bar.png}
        \caption[]{Gráfico de tiempo de acceso SDSL4Py en escala logarítmica para el input \textbf{yellow\_tripdata\_2015\_01}.}
        \label{fig:sdsl4py_access_time_comparison_plot_log_bar_3}
    \end{figure}
}

\DeclareRobustCommand{\SDSLFourPyCompressionSpaceComparisonOnePlotBar}{
    \begin{figure}[H]
        \centering
        \includegraphics[width=1\textwidth]{anexo/exp/SDSL4Py Compression Space Comparison/bar_plots/SDSL4Py Compression Space Comparison_d_08_1_1_2_log_bar.png}
        \caption[]{Gráfico de espacio de compresión SDSL4Py en escala logarítmica para el input \textbf{d\_08\_1\_1\_2}.}
        \label{fig:sdsl4py_compression_space_comparison_plot_log_bar_1}
    \end{figure}
}

\DeclareRobustCommand{\SDSLFourPyCompressionSpaceComparisonTwoPlotBar}{
    \begin{figure}[H]
        \centering
        \includegraphics[width=1\textwidth]{anexo/exp/SDSL4Py Compression Space Comparison/bar_plots/SDSL4Py Compression Space Comparison_d_08_1_1_10_log_bar.png}
        \caption[]{Gráfico de espacio de compresión SDSL4Py en escala logarítmica para el input \textbf{d\_08\_1\_1\_10}.}
        \label{fig:sdsl4py_compression_space_comparison_plot_log_bar_2}
    \end{figure}
}

\DeclareRobustCommand{\SDSLFourPyCompressionSpaceComparisonThreePlotBar}{
    \begin{figure}[H]
        \centering
        \includegraphics[width=1\textwidth]{anexo/exp/SDSL4Py Compression Space Comparison/bar_plots/SDSL4Py Compression Space Comparison_yellow_tripdata_2015-01_log_bar.png}
        \caption[]{Gráfico de espacio de compresión SDSL4Py en escala logarítmica para el input \textbf{yellow\_tripdata\_2015\_01}.}
        \label{fig:sdsl4py_compression_space_comparison_plot_log_bar_3}
    \end{figure}
}
\DeclareRobustCommand{\SDSLFourPyCompressionTimeComparisonOnePlotBar}{
    \begin{figure}[H]
        \centering
        \includegraphics[width=1\textwidth]{anexo/exp/SDSL4Py Compression Time Comparison/bar_plots/SDSL4Py Compression Time Comparison_d_08_1_1_2_log_bar.png}
        \caption[]{Gráfico de tiempo de compresión SDSL4Py en escala logarítmica para el input \textbf{d\_08\_1\_1\_2}.}
        \label{fig:sdsl4py_compression_time_comparison_plot_log_bar_1}
    \end{figure}
}

\DeclareRobustCommand{\SDSLFourPyCompressionTimeComparisonTwoPlotBar}{
    \begin{figure}[H]
        \centering
        \includegraphics[width=1\textwidth]{anexo/exp/SDSL4Py Compression Time Comparison/bar_plots/SDSL4Py Compression Time Comparison_d_08_1_1_10_log_bar.png}
        \caption[]{Gráfico de tiempo de compresión SDSL4Py en escala logarítmica para el input \textbf{d\_08\_1\_1\_10}.}
        \label{fig:sdsl4py_compression_time_comparison_plot_log_bar_2}
    \end{figure}
}

\DeclareRobustCommand{\SDSLFourPyCompressionTimeComparisonThreePlotBar}{
    \begin{figure}[H]
        \centering
        \includegraphics[width=1\textwidth]{anexo/exp/SDSL4Py Compression Time Comparison/bar_plots/SDSL4Py Compression Time Comparison_yellow_tripdata_2015-01_log_bar.png}
        \caption[]{Gráfico de tiempo de compresión SDSL4Py en escala logarítmica para el input \textbf{yellow\_tripdata\_2015\_01}.}
        \label{fig:sdsl4py_compression_time_comparison_plot_log_bar_3}
    \end{figure}
}


% Vega-Altair Plot Time Comparison
\DeclareRobustCommand{\VegaAltairPlotTimeComparisonOnePlotLine}{
    %insertar imagen
    \begin{figure}[H]
        \centering
        \includegraphics[width=1\textwidth]{anexo/exp/Vega-Altair Plot Time Comparison/plots/Vega-Altair Plot Time Comparison_d_08_1_1_2_linear_line.png}
        \caption[]{Gráfico de tiempo de renderización Vega-Altair para el input \textbf{d\_08\_1\_1\_2}.}
        \label{fig:vega_altair_plot_time_comparison_plot_line_1}
    \end{figure}
    \begin{figure}[H]
        \centering
        \includegraphics[width=1\textwidth]{anexo/exp/Vega-Altair Plot Time Comparison/plots/Vega-Altair Plot Time Comparison_d_08_1_1_2_log_line.png}
        \caption[]{Gráfico de tiempo de renderización Vega-Altair para el input \textbf{d\_08\_1\_1\_2} en escala logarítmica.}
        \label{fig:vega_altair_plot_time_comparison_plot_line_1}
    \end{figure}
}

\DeclareRobustCommand{\VegaAltairPlotTimeComparisonOnePlotBar}{
    %insertar imagen
    \begin{figure}[H]
        \centering
        \includegraphics[width=1\textwidth]{anexo/exp/Vega-Altair Plot Time Comparison/bar_plots/Vega-Altair Plot Time Comparison_d_08_1_1_2_linear_bar.png}
        \caption[]{Gráfico de tiempo de renderización Vega-Altair para el input \textbf{d\_08\_1\_1\_2}.}
        \label{fig:vega_altair_plot_time_comparison_plot_bar_1}
    \end{figure}
}

\DeclareRobustCommand{\VegaAltairPlotTimeComparisonTwoPlotLine}{
    %insertar imagen
    \begin{figure}[H]
        \centering
        \includegraphics[width=1\textwidth]{anexo/exp/Vega-Altair Plot Time Comparison/plots/Vega-Altair Plot Time Comparison_d_08_1_1_10_linear_line.png}
        \caption[]{Gráfico de tiempo de renderización Vega-Altair para el input \textbf{d\_08\_1\_1\_10}.}
        \label{fig:vega_altair_plot_time_comparison_plot_line_2}
    \end{figure}
}

\DeclareRobustCommand{\VegaAltairPlotTimeComparisonTwoPlotBar}{
    %insertar imagen
    \begin{figure}[H]
        \centering
        \includegraphics[width=1\textwidth]{anexo/exp/Vega-Altair Plot Time Comparison/bar_plots/Vega-Altair Plot Time Comparison_d_08_1_1_10_linear_bar.png}
        \caption[]{Gráfico de tiempo de renderización Vega-Altair para el input \textbf{d\_08\_1\_1\_10}.}
        \label{fig:vega_altair_plot_time_comparison_plot_bar_2}
    \end{figure}
}

\DeclareRobustCommand{\VegaAltairPlotTimeComparisonThreePlotLine}{
    %insertar imagen
    \begin{figure}[H]
        \centering
        \includegraphics[width=1\textwidth]{anexo/exp/Vega-Altair Plot Time Comparison/plots/Vega-Altair Plot Time Comparison_yellow_tripdata_2015-01_linear_line.png}
        \caption[]{Gráfico de tiempo de renderización Vega-Altair para el input \textbf{yellow\_tripdata\_2015\_01}.}
        \label{fig:vega_altair_plot_time_comparison_plot_line_3}
    \end{figure}
}

\DeclareRobustCommand{\VegaAltairPlotTimeComparisonThreePlotBar}{
    %insertar imagen
    \begin{figure}[H]
        \centering
        \includegraphics[width=1\textwidth]{anexo/exp/Vega-Altair Plot Time Comparison/bar_plots/Vega-Altair Plot Time Comparison_yellow_tripdata_2015-01_linear_bar.png}
        \caption[]{Gráfico de tiempo de renderización Vega-Altair para el input \textbf{yellow\_tripdata\_2015\_01}.}
        \label{fig:vega_altair_plot_time_comparison_plot_bar_3}
    \end{figure}
}

% Vega-Altair Plotting + Building Comparison
\DeclareRobustCommand{\VegaAltairPlottingBuildingComparisonOnePlotLine}{
    %insertar imagen
    \begin{figure}[H]
        \centering
        \includegraphics[width=1\textwidth]{anexo/exp/Vega-Altair Plotting + Building Comparison/plots/Vega-Altair Plotting + Building Comparison_d_08_1_1_2_linear_line.png}
        \caption[]{Gráfico de tiempo de renderización y construcción Vega-Altair para el input \textbf{d\_08\_1\_1\_2}.}
        \label{fig:vega_altair_plotting_building_comparison_plot_line_1}
    \end{figure}
}

\DeclareRobustCommand{\VegaAltairPlottingBuildingComparisonOnePlotBar}{
    %insertar imagen
    \begin{figure}[H]
        \centering
        \includegraphics[width=1\textwidth]{anexo/exp/Vega-Altair Plotting + Building Comparison/bar_plots/Vega-Altair Plotting + Building Comparison_d_08_1_1_2_linear_bar.png}
        \caption[]{Gráfico de tiempo de renderización y construcción Vega-Altair para el input \textbf{d\_08\_1\_1\_2}.}
        \label{fig:vega_altair_plotting_building_comparison_plot_bar_1}
    \end{figure}
}

\DeclareRobustCommand{\VegaAltairPlottingBuildingComparisonTwoPlotLine}{
    %insertar imagen
    \begin{figure}[H]
        \centering
        \includegraphics[width=1\textwidth]{anexo/exp/Vega-Altair Plotting + Building Comparison/plots/Vega-Altair Plotting + Building Comparison_d_08_1_1_10_linear_line.png}
        \caption[]{Gráfico de tiempo de renderización y construcción Vega-Altair para el input \textbf{d\_08\_1\_1\_10}.}
        \label{fig:vega_altair_plotting_building_comparison_plot_line_2}
    \end{figure}
}

\DeclareRobustCommand{\VegaAltairPlottingBuildingComparisonTwoPlotBar}{
    %insertar imagen
    \begin{figure}[H]
        \centering
        \includegraphics[width=1\textwidth]{anexo/exp/Vega-Altair Plotting + Building Comparison/bar_plots/Vega-Altair Plotting + Building Comparison_d_08_1_1_10_linear_bar.png}
        \caption[]{Gráfico de tiempo de renderización y construcción Vega-Altair para el input \textbf{d\_08\_1\_1\_10}.}
        \label{fig:vega_altair_plotting_building_comparison_plot_bar_2}
    \end{figure}
}

\DeclareRobustCommand{\VegaAltairPlottingBuildingComparisonThreePlotLine}{
    %insertar imagen
    \begin{figure}[H]
        \centering
        \includegraphics[width=1\textwidth]{anexo/exp/Vega-Altair Plotting + Building Comparison/plots/Vega-Altair Plotting + Building Comparison_yellow_tripdata_2015-01_linear_line.png}
        \caption[]{Gráfico de tiempo de renderización y construcción Vega-Altair para el input \textbf{yellow\_tripdata\_2015\_01}.}
        \label{fig:vega_altair_plotting_building_comparison_plot_line_3}
    \end{figure}
}

\DeclareRobustCommand{\VegaAltairPlottingBuildingComparisonThreePlotBar}{
    %insertar imagen
    \begin{figure}[H]
        \centering
        \includegraphics[width=1\textwidth]{anexo/exp/Vega-Altair Plotting + Building Comparison/bar_plots/Vega-Altair Plotting + Building Comparison_yellow_tripdata_2015-01_linear_bar.png}
        \caption[]{Gráfico de tiempo de renderización y construcción Vega-Altair para el input \textbf{yellow\_tripdata\_2015\_01}.}
        \label{fig:vega_altair_plotting_building_comparison_plot_bar_3}
    \end{figure}
}

% VEGA ALTAIR MEMORY ALLOCATION

\DeclareRobustCommand{\VegaAltairMemoryAllocationOnePlotLine}{
    %insertar imagen
    \begin{figure}[H]
        \centering
        \includegraphics[width=1\textwidth]{anexo/exp/Vega-Altair Plotting Memory Allocation/plots/Vega-Altair Plotting Memory Allocation_d_08_1_1_2_log_line.png}
        \caption[]{Gráfico de tiempo de renderización y construcción Vega-Altair para el input \textbf{d\_08\_1\_1\_2}.}
        \label{fig:vega_altair_plotting_building_comparison_plot_line_1}
    \end{figure}
}

\DeclareRobustCommand{\VegaAltairMemoryAllocationOnePlotBar}{
    %insertar imagen
    \begin{figure}[H]
        \centering
        \includegraphics[width=1\textwidth]{anexo/exp/Vega-Altair Plotting Memory Allocation/bar_plots/Vega-Altair Plotting Memory Allocation_d_08_1_1_2_log_bar.png}
        \caption[]{Gráfico de tiempo de renderización y construcción Vega-Altair para el input \textbf{d\_08\_1\_1\_2}.}
        \label{fig:vega_altair_plotting_building_comparison_plot_bar_1}
    \end{figure}
\begin{table}[H]
\centering
\caption{\textit{d\_08\_1\_1\_2} – Memoria por elemento [KB]}
\label{tab:Vega-Altair Plotting Memory Allocation\_d\_08\_1\_1\_2_mem_por_elemento_sin_original}
\begin{tabular}{rrrr}
\toprule
 & CVD - dac\_vector & CVD - vlc\_vector\_fibonacci & TS Downsample \\
n\_size &  &  &  \\
\midrule
6000 & 0.16 & 0.16 & 0.16 \\
12000 & 0.16 & 0.16 & 0.16 \\
20000 & 0.17 & 0.17 & 0.16 \\
70000 & 0.18 & 0.18 & 0.17 \\
100000 & 0.18 & 0.18 & 0.17 \\
300000 & 0.19 & 0.18 & 0.17 \\
\bottomrule
\end{tabular}

\end{table}
}

\DeclareRobustCommand{\VegaAltairMemoryAllocationTwoPlotLine}{
    %insertar imagen
    \begin{figure}[H]
        \centering
        \includegraphics[width=1\textwidth]{anexo/exp/Vega-Altair Plotting Memory Allocation/plots/Vega-Altair Plotting Memory Allocation_d_08_1_1_10_log_line.png}
        \caption[]{Gráfico de tiempo de renderización y construcción Vega-Altair para el input \textbf{d\_08\_1\_1\_10}.}
        \label{fig:vega_altair_plotting_building_comparison_plot_line_2}
    \end{figure}
}

\DeclareRobustCommand{\VegaAltairMemoryAllocationTwoPlotBar}{
    %insertar imagen
    \begin{figure}[H]
        \centering
        \includegraphics[width=1\textwidth]{anexo/exp/Vega-Altair Plotting Memory Allocation/bar_plots/Vega-Altair Plotting Memory Allocation_d_08_1_1_10_log_bar.png}
        \caption[]{Gráfico de asignación de memoria con la biblioteca Vega-Altair para el input \textbf{d\_08\_1\_1\_10}.}
        \label{fig:vega_altair_plotting_building_comparison_plot_bar_2}
    \end{figure}
\begin{table}[H]
\centering
\caption{\textit{d\_08\_1\_1\_10} – Memoria por elemento [KB]}
\label{tab:Vega-Altair Plotting Memory Allocation\_d\_08\_1\_1\_10_mem_por_elemento_sin_original}
\begin{tabular}{rrrr}
\toprule
 & CVD - dac\_vector & CVD - vlc\_vector\_fibonacci & TS Downsample \\
n\_size &  &  &  \\
\midrule
6000 & 0.16 & 0.16 & 0.16 \\
12000 & 0.16 & 0.16 & 0.16 \\
20000 & 0.17 & 0.17 & 0.16 \\
70000 & 0.18 & 0.18 & 0.17 \\
100000 & 0.18 & 0.18 & 0.17 \\
300000 & 0.18 & 0.20 & 0.17 \\
\bottomrule
\end{tabular}

\end{table}
}

\DeclareRobustCommand{\VegaAltairMemoryAllocationThreePlotLine}{
    %insertar imagen
    \begin{figure}[H]
        \centering
        \includegraphics[width=1\textwidth]{anexo/exp/Vega-Altair Plotting Memory Allocation/plots/Vega-Altair Plotting Memory Allocation_yellow_tripdata_2015-01_log_line.png}
        \caption[]{Gráfico de asignación de memoria con la biblioteca Vega-Altair para el input \textbf{yellow\_tripdata\_2015\_01}.}
        \label{fig:vega_altair_plotting_building_comparison_plot_line_3}
    \end{figure}
}

\DeclareRobustCommand{\VegaAltairMemoryAllocationThreePlotBar}{
    %insertar imagen
    \begin{figure}[H]
        \centering
        \includegraphics[width=1\textwidth]{anexo/exp/Vega-Altair Plotting Memory Allocation/bar_plots/Vega-Altair Plotting Memory Allocation_yellow_tripdata_2015-01_log_bar.png}
        \caption[]{Gráfico de asignación de memoria con la biblioteca Vega-Altair para el input \textbf{yellow\_tripdata\_2015\_01}.}
        \label{fig:vega_altair_plotting_building_comparison_plot_bar_3}
    \end{figure}
\begin{table}[H]
\centering
\caption{\textit{yellow\_tripdata\_2015-01} – Memoria por elemento [KB]}
\label{tab:Vega-Altair Plotting Memory Allocation\_yellow\_tripdata\_2015-01_mem_por_elemento_sin_original}
\begin{tabular}{rrrr}
\toprule
 & CVD - dac\_vector & CVD - vlc\_vector\_fibonacci & TS Downsample \\
n\_size &  &  &  \\
\midrule
6000 & 0.16 & 0.16 & 0.16 \\
12000 & 0.16 & 0.16 & 0.16 \\
20000 & 0.16 & 0.16 & 0.16 \\
70000 & 0.16 & 0.16 & 0.16 \\
100000 & 0.16 & 0.16 & 0.16 \\
300000 & 0.16 & 0.16 & 0.16 \\
\bottomrule
\end{tabular}

\end{table}
}

\DefineRobustCommand{\all_libraries_memory_allocation_plot_1_table}
{
    \input{anexo/All Libraries Memory Allocation/txts/All Libraries Memory Allocation_d_08_1_1_2_linear_line.tex}
}
\DeclareRobustCommand{\AllLibrariesMemoryAllocation}
{
    \AllLibrariesMemoryAllocationOnePlotBar
    \AllLibrariesMemoryAllocationOnePlotLine
    \AllLibrariesMemoryAllocationOneTable

    \AllLibrariesMemoryAllocationTwoPlotBar
    \AllLibrariesMemoryAllocationTwoPlotLine
    \AllLibrariesMemoryAllocationTwoTable

    \AllLibrariesMemoryAllocationThreePlotBar
    \AllLibrariesMemoryAllocationThreePlotLine
    \AllLibrariesMemoryAllocationThreeTable
}

\DeclareRobustCommand{\AllLibrariesTimeComparison}
{
    \AllLibrariesTimeComparisonOnePlotBar
    \AllLibrariesTimeComparisonOnePlotLine
    \AllLibrariesTimeComparisonOneTable

    \AllLibrariesTimeComparisonTwoPlotBar
    \AllLibrariesTimeComparisonTwoPlotLine
    \AllLibrariesTimeComparisonTwoTable
    
    \AllLibrariesTimeComparisonThreePlotBar
    \AllLibrariesTimeComparisonThreePlotLine
    \AllLibrariesTimeComparisonThreeTable
}

\DeclareRobustCommand{\BuildingTimeComparison}
{
    \BuildingTimeComparisonOnePlotBar
    \BuildingTimeComparisonOnePlotLine
    \BuildingTimeComparisonOneTable

    \BuildingTimeComparisonTwoPlotBar
    \BuildingTimeComparisonTwoPlotLine
    \BuildingTimeComparisonTwoTable
    
    \BuildingTimeComparisonThreePlotBar
    \BuildingTimeComparisonThreePlotLine
    \BuildingTimeComparisonThreeTable
}

\DeclareRobustCommand{\ComparisonOfSpaceUsed}
{
    \ComparisonOfSpaceUsedOnePlotBar
    \ComparisonOfSpaceUsedOnePlotLine
    \ComparisonOfSpaceUsedOneTable

    \ComparisonOfSpaceUsedTwoPlotBar
    \ComparisonOfSpaceUsedTwoPlotLine
    \ComparisonOfSpaceUsedTwoTable
    
    \ComparisonOfSpaceUsedThreePlotBar
    \ComparisonOfSpaceUsedThreePlotLine
    \ComparisonOfSpaceUsedThreeTable
}

\DeclareRobustCommand{\CVDDecimalPlacesAccessTimeComparison}
{
    \CVDDecimalPlacesAccessTimeComparisonOnePlotBar
    \CVDDecimalPlacesAccessTimeComparisonOnePlotLine
    \CVDDecimalPlacesAccessTimeComparisonOneTable

    \CVDDecimalPlacesAccessTimeComparisonTwoPlotBar
    \CVDDecimalPlacesAccessTimeComparisonTwoPlotLine
    \CVDDecimalPlacesAccessTimeComparisonTwoTable
    
    \CVDDecimalPlacesAccessTimeComparisonThreePlotBar
    \CVDDecimalPlacesAccessTimeComparisonThreePlotLine
    \CVDDecimalPlacesAccessTimeComparisonThreeTable
}

\DeclareRobustCommand{\CVDDecimalPlacesBuildTimeComparison}
{
    \CVDDecimalPlacesBuildTimeComparisonOnePlotBar
    \CVDDecimalPlacesBuildTimeComparisonOnePlotLine
    \CVDDecimalPlacesBuildTimeComparisonOneTable

    \CVDDecimalPlacesBuildTimeComparisonTwoPlotBar
    \CVDDecimalPlacesBuildTimeComparisonTwoPlotLine
    \CVDDecimalPlacesBuildTimeComparisonTwoTable
    
    \CVDDecimalPlacesBuildTimeComparisonThreePlotBar
    \CVDDecimalPlacesBuildTimeComparisonThreePlotLine
    \CVDDecimalPlacesBuildTimeComparisonThreeTable
}

\DeclareRobustCommand{\CVDDecimalPlacesSizeComparison}
{
    \CVDDecimalPlacesSizeComparisonOnePlotBar
    \CVDDecimalPlacesSizeComparisonOnePlotLine
    \CVDDecimalPlacesSizeComparisonOneTable

    \CVDDecimalPlacesSizeComparisonTwoPlotBar
    \CVDDecimalPlacesSizeComparisonTwoPlotLine
    \CVDDecimalPlacesSizeComparisonTwoTable
    
    \CVDDecimalPlacesSizeComparisonThreePlotBar
    \CVDDecimalPlacesSizeComparisonThreePlotLine
    \CVDDecimalPlacesSizeComparisonThreeTable
}

\DeclareRobustCommand{\PyGalMemoryAllocation}
{
    \PyGalMemoryAllocationOnePlotBar
    \PyGalMemoryAllocationOnePlotLine
    \PyGalMemoryAllocationOneTable

    \PyGalMemoryAllocationTwoPlotBar
    \PyGalMemoryAllocationTwoPlotLine
    \PyGalMemoryAllocationTwoTable
    
    \PyGalMemoryAllocationThreePlotBar
    \PyGalMemoryAllocationThreePlotLine
    \PyGalMemoryAllocationThreeTable
}

\DeclareRobustCommand{\PyGalPlotTime}
{
    \PyGalPlotTimeOnePlotBar
    \PyGalPlotTimeOnePlotLine
    \PyGalPlotTimeOneTable

    \PyGalPlotTimeTwoPlotBar
    \PyGalPlotTimeTwoPlotLine
    \PyGalPlotTimeTwoTable
    
    \PyGalPlotTimeThreePlotBar
    \PyGalPlotTimeThreePlotLine
    \PyGalPlotTimeThreeTable
}

\DeclareRobustCommand{\SDSLFourPyAccessTimeComparison}
{
    \SDSLFourPyAccessTimeComparisonOnePlotBar
    \SDSLFourPyAccessTimeComparisonOnePlotLine
    \SDSLFourPyAccessTimeComparisonOneTable

    \SDSLFourPyAccessTimeComparisonTwoPlotBar
    \SDSLFourPyAccessTimeComparisonTwoPlotLine
    \SDSLFourPyAccessTimeComparisonTwoTable
    
    \SDSLFourPyAccessTimeComparisonThreePlotBar
    \SDSLFourPyAccessTimeComparisonThreePlotLine
    \SDSLFourPyAccessTimeComparisonThreeTable
}

\DeclareRobustCommand{\SDSLFourPyCompressionSpaceComparison}
{
    \SDSLFourPyCompressionSpaceComparisonOnePlotBar
    \SDSLFourPyCompressionSpaceComparisonOnePlotLine
    \SDSLFourPyCompressionSpaceComparisonOneTable

    \SDSLFourPyCompressionSpaceComparisonTwoPlotBar
    \SDSLFourPyCompressionSpaceComparisonTwoPlotLine
    \SDSLFourPyCompressionSpaceComparisonTwoTable
    
    \SDSLFourPyCompressionSpaceComparisonThreePlotBar
    \SDSLFourPyCompressionSpaceComparisonThreePlotLine
    \SDSLFourPyCompressionSpaceComparisonThreeTable
}

\DeclareRobustCommand{\SDSLFourPyCompressionTimeComparison}
{
    \SDSLFourPyCompressionTimeComparisonOnePlotBar
    \SDSLFourPyCompressionTimeComparisonOnePlotLine
    \SDSLFourPyCompressionTimeComparisonOneTable

    \SDSLFourPyCompressionTimeComparisonTwoPlotBar
    \SDSLFourPyCompressionTimeComparisonTwoPlotLine
    \SDSLFourPyCompressionTimeComparisonTwoTable
    
    \SDSLFourPyCompressionTimeComparisonThreePlotBar
    \SDSLFourPyCompressionTimeComparisonThreePlotLine
    \SDSLFourPyCompressionTimeComparisonThreeTable
}

\DeclareRobustCommand{\VegaAltairPlotTimeComparison}
{
    \VegaAltairPlotTimeComparisonOnePlotBar
    \VegaAltairPlotTimeComparisonOnePlotLine
    \VegaAltairPlotTimeComparisonOneTable

    \VegaAltairPlotTimeComparisonTwoPlotBar
    \VegaAltairPlotTimeComparisonTwoPlotLine
    \VegaAltairPlotTimeComparisonTwoTable
    
    \VegaAltairPlotTimeComparisonThreePlotBar
    \VegaAltairPlotTimeComparisonThreePlotLine
    \VegaAltairPlotTimeComparisonThreeTable
}

\DeclareRobustCommand{\VegaAltairPlottingBuildingComparison}
{
    \VegaAltairPlottingBuildingComparisonOnePlotBar
    \VegaAltairPlottingBuildingComparisonOnePlotLine
    \VegaAltairPlottingBuildingComparisonOneTable

    \VegaAltairPlottingBuildingComparisonTwoPlotBar
    \VegaAltairPlottingBuildingComparisonTwoPlotLine
    \VegaAltairPlottingBuildingComparisonTwoTable
    
    \VegaAltairPlottingBuildingComparisonThreePlotBar
    \VegaAltairPlottingBuildingComparisonThreePlotLine
    \VegaAltairPlottingBuildingComparisonThreeTable
}

\chapter{Pruebas Sobre el Sistema}
\label{sec:pruebas-sistema}

Usando el framework descrito en la Sección~\ref{benchmarking}, se declararon distintos experimentos a partir de la plantilla base del repositorio. Cada uno de ellos evalúa distintos aspectos del sistema como el uso de memoria, tiempos de ejecución o el tamaño ocupado por diferentes representaciones de datos. La descripción de cada experimento, así como sus valores de entrada y salida se encuentran a continuación.

\section{Complejidad Teórica}

El proceso completo de visualización basado en estructuras de datos compactas puede dividirse en tres etapas principales, cada una con una complejidad temporal asociada:

\begin{enumerate}
    \item \textbf{Downsampling:} \\
    Esta etapa reduce el conjunto de datos desde un tamaño original \(n\) hasta una submuestra de tamaño \(n_{\text{out}}\), seleccionando los puntos más representativos. De acuerdo con los autores de \texttt{tsdownsample}~\cite{tsdownsample}, esta operación tiene una complejidad temporal de \(\mathcal{O}(n)\).

    \item \textbf{Construcción de la estructura de datos compacta:} \\
    Una vez obtenido el subconjunto reducido, se codifica utilizando una estructura compacta con el objetivo de reducir el espacio ocupado en memoria. A pesar de la disponibilidad de tres tipos de vectores en \texttt{CompressedVectorDownsampler}, se omite el uso de \texttt{enc\_vector} por las limitaciones en su implementación, mencionadas en el Anexo\ref{limitaciones_enc_vector}.

    \begin{itemize}
        \item \textbf{\texttt{vlc\_vector}:} \\[0.5em]
            Este vector utiliza codificación de longitud variable para representar cada valor \(v_i\), transformándolo internamente en \(v_i + 1\) para garantizar que el valor a codificar sea mayor que cero. Luego, se aplica un codificador autocontenible —en este caso, \textit{Fibonacci}— que genera un \textit{codeword} cuya longitud es proporcional al número de términos en la representación de Zeckendorf de \(v_i\). La codificación sigue los siguientes pasos, con sus respectivas complejidades teóricas:
            
            \begin{enumerate}
                \item \textbf{Transformación de entrada:} \\
                Cada elemento \(v_i\) se transforma en \(v_i + 1\). Esta operación se realiza en tiempo constante, por lo que el costo total es \(\mathcal{O}(n_{\text{out}})\).
            
                \item \textbf{Cálculo del tamaño del bitstream:} \\
                Se recorre la submuestra y para cada valor se evalúa la longitud del codeword Fibonacci. Esta longitud es proporcional a \(\lceil\log_\varphi v_i\rceil\), donde \(\varphi\) es la razón áurea. Por lo tanto, la complejidad total es \(\mathcal{O}(n_{\text{out}} \cdot \log V)\), donde \(V = \max_i v_i\).
            
                \item \textbf{Codificación de los valores:} \\
                Cada valor es codificado y escrito secuencialmente en el bitstream \texttt{m\_z}. Este paso también tiene complejidad \(\mathcal{O}(\log v_i)\) por elemento, resultando en un total de \(\mathcal{O}(n_{\text{out}} \cdot \log V)\).
                
                \item \textbf{Inserción de punteros de muestreo:} \\
                Cada \(t\) elementos se guarda una posición de referencia en \texttt{m\_sample\_pointer} para permitir acceso aleatorio eficiente. Como esto se realiza \(\left\lceil \frac{n_{\text{out}}}{t} \right\rceil\) veces, la complejidad total es \(\mathcal{O}(n_{\text{out}})\) dado que \(t\) es constante.
            \end{enumerate}
            
            \noindent Por lo tanto, la complejidad total de construcción del vector comprimido es:
            \[
            \boxed{\mathcal{O}(n_{\text{out}} \cdot \log V)}
            \]
        
        \item \textbf{\texttt{dac\_vector}:} \\[0.5em]
            Esta estructura divide cada valor \(v_i\) en fragmentos de tamaño fijo \(b\) bits (por ejemplo, \(b = 4\) o \(b = 8\)). Cada fragmento se almacena en un nivel distinto, comenzando por el menos significativo, y se utilizan bits de control en un vector auxiliar para marcar si hay niveles adicionales. El objetivo es equilibrar compresión y acceso rápido.
            
            La construcción se realiza en:
            \[
            \boxed{\mathcal{O}\left(n \cdot \frac{\log V}{b}\right)}
            \]
            tal como en la versión original, pues el método de particionado en niveles no depende del codificador VLC.
        
        \item \textbf{Complejidad de decodificación}
        
            \begin{itemize}
                \item \textbf{\texttt{vlc\_vector} (Fibonacci):} \\
                    Los valores están codificados con \textit{Fibonacci} y almacenados secuencialmente en \texttt{m\_z}. Para acceso aleatorio, cada \(t\) elementos se guarda un puntero en \texttt{m\_sample\_pointer}.
                    
                    La decodificación de un elemento \(i\) implica:
                    \begin{enumerate}
                        \item Determinar el puntero base en \(\mathcal{O}(1)\).
                        \item Decodificar secuencialmente hasta \(i \bmod t\) valores desde dicho puntero, con complejidad \(\mathcal{O}(t \cdot \log_\varphi V)\).
                    \end{enumerate}
                    
                    Dado que \(t\) es constante:
                    \[
                    \boxed{\mathcal{O}(\log V)}
                    \]
                
                \item \textbf{\texttt{dac\_vector}:} \\
                    Se mantiene igual a la versión original, con:
                    \[
                    \boxed{\mathcal{O}\left(\frac{\log V}{b}\right)}
                    \]
            \end{itemize}
        
        \item \textbf{Renderizado desde la estructura comprimida:} \\
            Para \texttt{vlc\_vector} con \textit{Fibonacci}, el costo total es:
            \[
            \mathcal{O}(n_{\text{out}} \cdot \log_\varphi V)
            \]
            y para \texttt{dac\_vector}:
            \[
            \mathcal{O}\left(n_{\text{out}} \cdot \frac{\log V}{b}\right)
            \]
    \end{itemize}
\end{enumerate}


\noindent En resumen, el pipeline completo presenta la siguiente complejidad total, según la estructura compacta utilizada:

\begin{itemize}
    \item \textbf{Usando \texttt{vlc\_vector}:}
    \[
    \underbrace{\mathcal{O}(n)}_{\text{downsampling}} +
    \underbrace{\mathcal{O}(n_{\text{out}} \cdot \log V)}_{\text{compresión}} +
    \underbrace{\mathcal{O}(n_{\text{out}} \cdot \log V)}_{\text{renderizado}}
    \]
    
    En términos asintóticos, dado que \(n_{\text{out}} \ll n\), la complejidad total sigue siendo \(\mathcal{O}(n)\), dominada por el paso de \textit{downsampling}. Las otras etapas tienen un costo menor proporcional al tamaño reducido y logarítmico en el valor máximo.

    \item \textbf{Usando \texttt{dac\_vector}:}
    \[
    \underbrace{\mathcal{O}(n)}_{\text{downsampling}} +
    \underbrace{\mathcal{O}\left(n_{\text{out}} \cdot \frac{\log V}{b}\right)}_{\text{compresión}} +
    \underbrace{\mathcal{O}\left(n_{\text{out}} \cdot \frac{\log V}{b}\right)}_{\text{renderizado}}
    \]

    De forma similar, como \(b\) es una constante (por ejemplo, \(b = 4\) u \(8\)) y \(n_{\text{out}} \ll n\), la complejidad total también es \(\mathcal{O}(n)\). Sin embargo, \texttt{dac\_vector} presenta una ventaja teórica en el coeficiente constante al distribuir la codificación por niveles.
\end{itemize}

\bigskip

\noindent Es importante distinguir entre dos escenarios:

\begin{itemize}
    \item \textbf{Construcción y visualización inicial:} \\
    Este caso ocurre la primera vez que se desea visualizar un conjunto de datos. Se requiere realizar el \textit{downsampling}, construir la estructura compacta, y luego renderizar. Por tanto, la complejidad corresponde al total del pipeline anteriormente descrito.

    \item \textbf{Renderizaciones subsiguientes:} \\
    Una vez construida la estructura comprimida, se pueden generar nuevas visualizaciones sin repetir el proceso completo. Por ejemplo, si se desea re-renderizar el gráfico con otro estilo o color, solo se requiere decodificar los valores. En este caso, la complejidad se reduce únicamente al costo de renderizado:
    \begin{itemize}
        \item \texttt{vlc\_vector}: \(\mathcal{O}(n_{\text{out}} \cdot \log V)\)
        \item \texttt{dac\_vector}: \(\mathcal{O}\left(n_{\text{out}} \cdot \frac{\log V}{b}\right)\)
    \end{itemize}
\end{itemize}

\begin{table}[H]
\centering
\caption{Comparación de complejidades teóricas para \texttt{CompressedVectorDownsample}.}
\label{tabla:complejidades_vectores}
\begin{tabular}{@{}lcc@{}}
\toprule
\textbf{Operación} & \textbf{\texttt{vlc\_vector}} & \textbf{\texttt{dac\_vector}} \\
\midrule
Downsampling                           & $\mathcal{O}(n)$                                     & $\mathcal{O}(n)$ \\[0.5em]
Construcción de estructura             & $\mathcal{O}(n_{\text{out}} \cdot \log V)$           & $\mathcal{O}(n_{\text{out}} \cdot \frac{\log V}{b})$ \\[0.5em]
Decodificación de un elemento          & $\mathcal{O}(\log V)$                                & $\mathcal{O}(\frac{\log v}{b})$ \\[0.5em]
Renderizado completo                   & $\mathcal{O}(n_{\text{out}} \cdot \log V)$           & $\mathcal{O}(n_{\text{out}} \cdot \frac{\log V}{b})$ \\[0.5em]
\bottomrule
\end{tabular}
\end{table}

\newpage
\section{Información General}
\label{general_info}

\paragraph{Equipo de Pruebas}
Los experimentos fueron ejecutados en el siguiente entorno de pruebas:

Todos los experimentos se realizaron en el servidor \texttt{chome} de la Universidad de Concepción, el cual fue proporcionado específicamente para el desarrollo de esta memoria de título. 

\begin{itemize}
    \item \textbf{Hostname:} chome
    \item \textbf{CPU:} Intel(R) Xeon(R) Gold 5320T CPU @ 2.30GHz
    \item \textbf{Sistema Operativo:} Linux 5.10.0-13-amd64-x86\_64-with-glibc2.31
    \item \textbf{Versión de Python:} 3.9.2
    \item \textbf{Dependencias principales para el framework de experimentos:}
    \begin{itemize}
        \item numpy==2.0.2
            % explicar para que se usa
            Usada para calcular promedio, desviación estándar, entre otros.
        \item sacred==0.8.7
            % explicar para que se usa
            Usada para definir y ejecutar los experimentos.
    \end{itemize}
    \item \textbf{Directorio base:} /home/obrito2020/oliver/cv\_visualization/benchmarking
    \item \textbf{Repositorio:} git@github.com:rat00lis/cv\_visualization.git
    \item \textbf{Commit:} \texttt{8bf12c009fc991b9eeb2cb655b4d5cc3b5030a44}
\end{itemize}

\paragraph{Configuración}

\begin{itemize}
    \item \textbf{Iteraciones por experimento:} $100$
    \item \textbf{Rangos de entrada:} $(6000, 12000, 20000, 70000, 100000, 300000)$
\end{itemize}

\paragraph{Datos de Entrada}

\begin{itemize}
    \item \textbf{Sensores de Puente:} Conjunto de datos en formato tabular, del cual se utiliza la columna 1 como eje \(y\). Esta columna contiene valores en punto flotante con hasta 4 posiciones decimales, que varían aproximadamente entre \(-1\) y \(1\). Estos datos emulan la señal capturada por sensores instalados en puentes. Este \textit{dataset} fue proporcionado por el profesor guía Gonzalo Rojas.
    \label{input_puentes}

        Para el primer conjunto de datos, se cuenta con un total de $m = 13$ archivos. El objetivo es determinar qué tan diferentes son entre sí en cuanto a su variabilidad de rango, para evaluar si la elección de uno u otro influiría significativamente en los resultados de las pruebas.
        
        El script aplicado realiza el siguiente procedimiento para cada archivo:
        \begin{enumerate}
            \item Define un conjunto de puntos de evaluación $P = \{6000, 12000, 20000, 70000, 100000, 300000\}$.
            \item Para cada $p_i \in P$, calcula el \textbf{rango} de los datos desde el inicio hasta $p_i$:
            \[
            R(p_i) = \max(X[1:p_i]) - \min(X[1:p_i])
            \]
            \item Obtiene métricas estadísticas:
            \begin{itemize}
                \item \textbf{Coeficiente de variación} (CV), que mide la variabilidad relativa.
                \item \textbf{Tasa de crecimiento} del rango (GR), que mide cuánto aumenta el rango a lo largo de los puntos de evaluación.
                \item \textbf{Índice de similitud} (IS), que indica qué tan similares son los patrones entre archivos.
            \end{itemize}
        \end{enumerate}
        
        El resultado para este conjunto de datos fue:
        \begin{itemize}
            \item Número de archivos analizados: 13.
            \item Coeficiente de variación promedio: $0.8039 \pm 0.2029$ (rango: $0.3997$ a $1.1348$).
            \item Tasa de crecimiento promedio: $28.3641 \pm 21.5020$ (rango: $1.7979$ a $55.1019$).
            \item Índice de similitud: $0.2524$.
        \end{itemize}
        
        Esto significa que los archivos presentan \textbf{patrones similares} de variabilidad de rango, por lo que la elección de uno para el experimento no afectaría significativamente los resultados.
        
        Sin embargo, para maximizar el rango observado, revisamos la tabla de resultados y calculamos la \textbf{diferencia máxima} por archivo (máximo de $R(p_i)$ en todos los puntos de evaluación):
        
        \begin{table}[H]
        \centering
        \begin{tabular}{lrrrrrrr}
        \hline
        \textbf{File} & @6000 & @12000 & @20000 & @70000 & @100000 & @300000  \\
        \hline
        d\_08\_1\_1\_1  & 0.99 & 1.47 & 1.47 & 1.47 & 8.37 & 12.18  \\
        d\_08\_1\_1\_10 & 0.68 & 0.68 & 3.91 & 4.98 & 4.98 & 5.65   \\
        d\_08\_1\_1\_11 & 1.14 & 1.14 & 4.32 & 4.32 & 4.39 & 8.09  \\
        d\_08\_1\_1\_12 & 1.36 & 1.36 & 1.36 & 4.63 & 8.23 & 9.63   \\
        d\_08\_1\_1\_13 & 2.26 & 2.26 & 3.23 & 3.23 & 3.23 & 6.33   \\
        d\_08\_1\_1\_2  & 0.26 & 0.26 & 0.36 & 6.17 & 6.17 & 12.36  \\
        d\_08\_1\_1\_3  & 0.31 & 1.74 & 4.52 & 4.72 & 9.12 & 9.12   \\
        d\_08\_1\_1\_4  & 0.27 & 0.27 & 0.27 & 8.65 & 8.65 & 8.65    \\
        d\_08\_1\_1\_5  & 0.21 & 0.53 & 2.62 & 5.26 & 11.56 & 11.56 \\
        d\_08\_1\_1\_6  & 0.21 & 0.53 & 2.62 & 5.26 & 11.56 & 11.56  \\
        d\_08\_1\_1\_7  & 0.21 & 0.53 & 2.62 & 5.26 & 11.56 & 11.56  \\
        d\_08\_1\_1\_8  & 0.21 & 0.53 & 2.62 & 5.26 & 11.56 & 11.56 \\
        d\_08\_1\_1\_9  & 0.65 & 1.77 & 1.77 & 5.54 & 6.51 & 6.51 \\
        \hline
        \end{tabular}
        \caption{Rangos por cada archivo de \textit{dataset} 1.}
        \end{table}
        
        Con base en esta revisión, se selecciona \textbf{d\_08\_1\_1\_10} y \textbf{d\_08\_1\_1\_2}, ya que presentan la diferencia menor y mayor dentro de los archivos disponibles.
        \begin{itemize}
            \item \texttt{d\_08\_1\_1\_2.txt}
            \item \texttt{d\_08\_1\_1\_10.txt}
        \end{itemize}
    
    \item \textbf{Usuarios por fecha en Taxi New York:} Conjunto de datos con marcas de tiempo en formato UNIX (columna \(x\)) y cantidad promedio de pasajeros transportados (columna 4, utilizada como eje \(y\)). Todos los valores son enteros positivos, sin decimales.\cite{kaggle_taxi}

    La intención de escoger este \textit{dataset} para contrastar con el previamente explicado, es debido a que es un tipo de información que no requiere decimales y que tiene poca variabilidad entre sí. Por ende, la compactación del vector \texttt{CompressedVector} es mucho menor, ya que no necesita construir un vector para almacenar los decimales.
    
    \label{input_taxi}
        \begin{itemize}
            \item \texttt{yellow\_tripdata\_2015\_01.txt}
        \end{itemize}
\end{itemize}


\newpage
\section{Tiempo}

\subsection{Comparación de Tiempo de Visualización}

\setcounter{secnumdepth}{3}
\setcounter{tocdepth}{3}
\subsubsection{All Libraries Time Comparison}
\label{all_libraries_time_comparison}

Experimento que compara el tiempo de ejecución para generar gráficos utilizando diferentes bibliotecas de visualización de datos. Se mide el tiempo total desde el inicio de la creación del gráfico hasta su finalización.

\paragraph{Entrada}
\begin{itemize}
    \item Conjunto de datos de entrada: \( (x_1, y_1), (x_2, y_2), \ldots, (x_n, y_n) \)
    \item Biblioteca a utilizar: Vega-Altair, Plotly, Pygal o Matplotlib.
\end{itemize}

\paragraph{Salida}
\begin{itemize}
    \item Tiempo de ejecución en segundos para generar el gráfico.
\end{itemize}

\newpage
\paragraph{Resultados}
\vspace{0.5em}
\noindent
\AllLibrariesTimeComparison
\newpage


\subsubsection{Vega-Altair Plot Time Comparison}
\label{vega_altair_plot_time}

Experimento específico que mide el tiempo de generación de gráficos utilizando únicamente la biblioteca Vega-Altair con diferentes configuraciones y tamaños de datos.

\paragraph{Entrada}
\begin{itemize}
    \item Conjunto de datos de entrada: \( (x_1, y_1), (x_2, y_2), \ldots, (x_n, y_n) \)
    \item Configuraciones específicas de Vega-Altair para diferentes tipos de gráficos.
\end{itemize}

\paragraph{Salida}
\begin{itemize}
    \item Tiempo de ejecución detallado para cada configuración de Vega-Altair.
\end{itemize}

\newpage
\paragraph{Resultados}
\vspace{0.5em}
\noindent
\VegaAltairPlotTimeComparison
\newpage

\subsubsection{PyGal Plot Time}
\label{pygal_plot_time}

Experimento que evalúa el rendimiento temporal de la biblioteca PyGal para la generación de diferentes tipos de gráficos.

\paragraph{Entrada}
\begin{itemize}
    \item Conjunto de datos de entrada: \( (x_1, y_1), (x_2, y_2), \ldots, (x_n, y_n) \)
    \item Configuraciones específicas de PyGal.
\end{itemize}

\paragraph{Salida}
\begin{itemize}
    \item Tiempo de ejecución para generar gráficos con PyGal.
\end{itemize}

\newpage
\paragraph{Resultados}
\vspace{0.5em}
\noindent
\PyGalPlotTime
\newpage

Los resultados obtenidos en los experimentos~\ref{vega_altair_plot_time} y~\ref{pygal_plot_time} demuestran que el tiempo de renderizado disminuye considerablemente al aplicar técnicas de \textit{downsampling}, en comparación con la visualización directa del conjunto original sin preprocesamiento. Tanto \texttt{TSDownsample} como \texttt{CompressedVectorDownsampler} logran este objetivo al reducir la cantidad de puntos a representar, lo que permite acelerar la visualización sin sacrificar interpretabilidad.

Al comparar entre ambos métodos, se observa que \texttt{CompressedVectorDownsampler} presenta tiempos de renderizado consistentemente mayores que \texttt{TSDownsample}, a pesar de producir listas con la misma cantidad de puntos \(n_{\text{out}}\). Esta diferencia se explica por el uso de estructuras de datos comprimidas, que introducen un costo adicional al momento de acceder a los valores.

En general, el tiempo total de renderizado está determinado por la cantidad de puntos \(n_{\text{out}}\) y la complejidad de decodificación \(D\) asociada a la estructura de datos utilizada. En otras palabras, la complejidad del proceso puede expresarse como \(\mathcal{O}(n_{\text{out}} \cdot D)\), donde \(D\) depende del tipo de representación.

\begin{itemize}
    \item En estructuras lineales como listas de Python o arreglos de NumPy, se tiene \(D = \mathcal{O}(1)\), ya que el acceso a los elementos es directo y constante.
    
    \item En estructuras de datos compactas como \texttt{vlc\_vector}, se utiliza una decodificación con complejidad \(D = \mathcal{O}(t \cdot \log v_i)\), donde \(t\) es la densidad de muestreo y \(v_i\) el valor a decodificar.
    
    \item En el caso de \texttt{dac\_vector}, se accede a cada valor con una complejidad de decodificación \(D = \mathcal{O}(\log_b v_i)\), según el número de niveles requeridos para reconstruir el entero \(v_i\).
\end{itemize}

En consecuencia, si bien \texttt{CompressedVectorDownsampler} permite representar la misma cantidad de puntos que \texttt{TSDownsample}, el tiempo de renderizado es mayor debido al costo de decodificación inherente a las estructuras utilizadas. 




\newpage
\subsection{Comparación de Tiempo de Visualización + Construcción}
\subsubsection{Vega-Altair Plotting + Building Comparison}
\label{vega_altair_plot_plus_build_time}

Experimento que mide el tiempo total incluyendo tanto la generación del gráfico como su construcción/renderizado final utilizando Vega-Altair.

\paragraph{Entrada}
\begin{itemize}
    \item Conjunto de datos de entrada: \( (x_1, y_1), (x_2, y_2), \ldots, (x_n, y_n) \)
    \item Configuraciones de Vega-Altair para diferentes tipos de visualización.
\end{itemize}

\paragraph{Salida}
\begin{itemize}
    \item Tiempo total de plotting y building combinados.
\end{itemize}

\newpage
\paragraph{Resultados}
\vspace{0.5em}
\noindent
\VegaAltairPlottingBuildingComparison
\newpage

A pesar de tener resultados positivos en la acción aislada de visualizar, es importante analizar el tiempo adicional de construcción de \texttt{CompressedVectorDownsampler} para efectos de analizar los resultados de su implementación.

Recordando entonces, las complejidades teóricas de procesar los datos originales y renderizarlos con \texttt{CompressedVectorDownsampler} y \texttt{TS Downsample}:

\begin{itemize}
    \item El tiempo total de preprocesamiento y visualización con \texttt{TSDownsample} es:
    \[
    \mathcal{O}(n) + \mathcal{O}(n_{\text{out}})
    \]
    gracias al acceso directo en memoria.

    \item En cambio, el tiempo total con \texttt{CompressedVectorDownsampler} es:
    \[
    \mathcal{O}(n) + \mathcal{O}(n_{\text{out}} \cdot \log V) + \mathcal{O}(n_{\text{out}} \cdot t \cdot \log V)
    \]

\end{itemize}

En el gráfico del experimento~\ref{vega_altair_plot_plus_build_time}, se puede observar que el tiempo de renderizado del resultado entregado por \texttt{CompressedVectorDownsampler} puede ser incluso mayor que el de los datos originales. Esto se debe, en parte, al costo del preprocesamiento necesario para construir los vectores, que no es despreciable, y al hecho de que cada par ordenado se representa internamente mediante \textbf{tres vectores compactos}, triplicando así el trabajo de compresión.

No obstante, a medida que crece la cantidad de puntos a procesar, el enfoque basado en estructuras de datos compactas alcanza tiempos más rápidos de visualización que los datos sin procesar. Esto se puede evidenciar para $N = 300,000$, donde \texttt{CompressedVectorDownsampler} se posiciona por debajo de la curva de los datos originales. 

Adicionalmente, se puede visualizar una mejora significativa para el dataset de entrada \textit{yellow\_tripdata\_2015\_01.txt}, que se mantiene por debajo de la curva de los datos originales en todo momento. Esto es debido a que no posee espacios decimales, permitiendo que sólo se construya un único vector por cada axis del gráfico.

\subsection{Análisis}

A pesar de que \texttt{CompressedVectorDownsampler} introduce una penalización en el tiempo de acceso, esta se compensa por la mejora significativa en el uso de memoria y la reducción en el volumen total de datos procesados. Al trabajar sobre una muestra reducida (\(n_{\text{out}} \ll n\)) y con estructuras comprimidas, el sistema logra mantener tiempos de visualización razonables sin comprometer la interpretabilidad de la información.

En definitiva, si bien la visualización con estructuras comprimidas más uso de downsampler no alcanza la velocidad de acceso de listas nativas, sigue representando una mejora considerable frente a la visualización directa de los datos originales. 


\newpage
\section{Espacio}

\subsection{Comparación de Espacio Utilizado}
\subsubsection{Comparison of Space Used}
\label{comparison_of_space_used}

Experimento que compara el espacio utilizado en memoria y almacenamiento por las diferentes bibliotecas durante el proceso de visualización de datos.

\paragraph{Entrada}
\begin{itemize}
    \item Conjunto de datos de entrada: \( (x_1, y_1), (x_2, y_2), \ldots, (x_n, y_n) \)
    \item Biblioteca a utilizar: Vega-Altair, Plotly, Pygal o Matplotlib.
\end{itemize}

\paragraph{Salida}
\begin{itemize}
    \item Espacio utilizado en bytes por cada biblioteca.
\end{itemize}

\newpage
\paragraph{Resultados}
\vspace{0.5em}
\noindent
\ComparisonOfSpaceUsed
\newpage

Como se observa en los experimentos~\ref{comparison_of_space_used}, el uso de \texttt{CompressedVectorDownsampler} permite una reducción significativa del espacio en disco ocupado, en comparación con representaciones tradicionales. Aunque \texttt{TSDownsample} y \texttt{CompressedVectorDownsampler} generan la misma cantidad de puntos \(n_{\text{out}}\), la estructura interna de esta última codifica los valores mediante estructuras compactas, lo que resulta en archivos de menor tamaño.

Sin embargo, al analizar los conjuntos de datos de sensores de puentes~\ref{input_puentes}, se observa que el espacio ocupado por \texttt{CompressedVectorDownsampler} crece a medida de aumentar el tamaño de entrada, a pesar de que los valores de salida son siempre $n_{out}$. Este fenómeno está directamente relacionado con el rango creciente de valores presentes en la secuencia, como se muestra en la Figura~\ref{fig:range}.

Ambas estructuras utilizadas en este contexto —\texttt{vlc\_vector\_fibonnaci} y \texttt{dac\_vector}— dependen del tamaño de los valores a codificar: en el caso de \texttt{vlc\_vector\_fibonnaci}, la longitud del código binario crece con el valor representado; y en \texttt{dac\_vector}, el número de niveles requeridos para almacenar un entero también aumenta con su magnitud. Por lo tanto, a medida que se acumulan valores de mayor amplitud, las estructuras deben almacenar más bits por entrada, lo que lleva a un crecimiento del tamaño del archivo incluso si la cantidad de puntos permanece constante.

\begin{figure}[H]
    \centering
    \includegraphics[width=0.8\linewidth]{anexo//exp//Comparison of Space Used//plots/range_acumulated.png}
    \caption[Rango de Valores Acumulado Para Dataset \textit{d\_08\_1\_1\_10.txt}]{Diferencia entre el valor máximo y mínimo desde el punto $y_0$ al $y_{n_i}$ \textit{d\_08\_1\_1\_10.txt}}
    \label{fig:range}
\end{figure}

Otra anomalía observada en la visualización corresponde al conjunto de datos \textit{yellow\_tripdata\_2015-01}, donde el \texttt{dac\_vector} ocupa más espacio que \texttt{TSDownsample}. Una posible explicación es que las marcas de tiempo requieren un gran número de niveles en este esquema de compactación, ya que para representarlas en formato binario se necesitan al menos 32 bits. Adicionalmente, aunque se eliminen los decimales, la estructura interna sigue almacenando un vector para indicar el signo de cada número; y al utilizar un mismo ancho para todos los valores, termina ocupando más espacio en disco del necesario. Este comportamiento sugiere que un ajuste en el manejo del ancho de los elementos de cada vector (de manera que se ocupen anchos distintos para cada parte de la clase \texttt{CompressedVector}) podría reducir significativamente el tamaño final del archivo. Este acercamiento se explora en las proyecciones\ref{proyecciones}


\newpage
\subsection{Asignación de Memoria}

La \textbf{asignación de memoria} se refiere a la cantidad de memoria RAM que un programa reserva para almacenar y manipular datos mientras se está ejecutando.  
A diferencia del espacio utilizado en disco, que mide el tamaño final de los datos almacenados, la asignación de memoria refleja el uso temporal de recursos durante el procesamiento, incluyendo:
\begin{itemize}
    \item Los datos en sí mismos.
    \item Estructuras intermedias creadas durante cálculos o transformaciones.
    \item Buffers y metadatos necesarios para la visualización.
\end{itemize}

Es importante medir este uso de memoria para asegurar que no se introduce un uso excesivo de memoria al acceder a los datos compactos y renderizarlos.

\subsubsection{All Libraries Memory Allocation}
\label{all_libraries_memory_allocation}

%descripcion
Experimento que mide la memoria asignada por las diferentes bibliotecas al momento de crear un gráfico. Se incluye la memoria asignada al descomprimir los datos de entrada si la biblioteca no es compatible con \texttt{CompressedVector}.

\paragraph{Entrada}
%lista
\begin{itemize}
    \item Conjunto de datos de entrada: \( (x_1, y_1), (x_2, y_2), \ldots, (x_n, y_n) \)
    \item Biblioteca a utilizar: Vega-Altair, Plotly, Pygal o Matplotlib.
\end{itemize}

\paragraph{Salida}
%lista
\begin{itemize}
    \item Memoria asignada en el proceso de creación del gráfico.
\end{itemize}

\newpage
\paragraph{Resultados}
\vspace{0.5em}
\noindent
\AllLibrariesMemoryAllocation
\newpage


\subsubsection{Vega-Altair Memory Allocation}
\label{vega_altair_memory_allocation}

Experimento específico que analiza la asignación de memoria de la biblioteca Vega-Altair durante la creación de gráficos.

\paragraph{Entrada}
\begin{itemize}
    \item Conjunto de datos de entrada: \( (x_1, y_1), (x_2, y_2), \ldots, (x_n, y_n) \)
    \item Configuraciones específicas de Vega-Altair.
\end{itemize}

\paragraph{Salida}
\begin{itemize}
    \item Memoria asignada por Vega-Altair durante el proceso de visualización.
\end{itemize}

\newpage
\paragraph{Resultados}
\vspace{0.5em}
\noindent
\VegaAltairMemoryAllocation
\newpage


\subsubsection{PyGal Memory Allocation}
\label{pygal_memory_allocation}

Experimento específico que analiza la asignación de memoria de la biblioteca PyGal durante la creación de gráficos.

\paragraph{Entrada}
\begin{itemize}
    \item Conjunto de datos de entrada: \( (x_1, y_1), (x_2, y_2), \ldots, (x_n, y_n) \)
    \item Configuraciones específicas de PyGal.
\end{itemize}

\paragraph{Salida}
\begin{itemize}
    \item Memoria asignada por PyGal durante el proceso de visualización.
\end{itemize}

\newpage
\paragraph{Resultados}
\vspace{0.5em}
\noindent
\PyGalMemoryAllocation
\newpage

En los experimentos~\ref{vega_altair_memory_allocation} y~\ref{pygal_memory_allocation} se evalúa la memoria asignada durante la visualización de los datos utilizando distintas bibliotecas. Para este análisis, se toma como referencia una \textit{representación tradicional}, es decir, un arreglo plano de \(n\) puntos sin aplicar compresión ni reducción (downsampling), cargado directamente en memoria como listas o arrays comunes.

La complejidad asintótica del uso de memoria en ambos casos se mantiene en \(\mathcal{O}(n_{\text{out}})\), donde \(n_{\text{out}}\) corresponde al número de puntos seleccionados para la visualización. Sin embargo, no se ve una mejora del uso de memoria para el proceso de renderización, a pesar de que las estructuras de datos compactas utilizan menos espacio en disco, esto no se refleja a la hora de la visualización.

A pesar de que estas bibliotecas nos permiten iterar por sobre los datos compactos, aún así crean estructuras intermedias para acceder a los valores, usando una memoria muy cercana a la de los resultados del downsampling. Es por eso que no se encuentra ninguna ventaja respecto al uso de estructuras de datos compactas para la reducción de la memoria utilizada.

\subsection{Análisis}

El uso de estructuras compactas como \texttt{CompressedVectorDownsampler} permite reducir de forma efectiva el espacio en disco ocupado por los datos visualizables, lo cual representa una ventaja clara en términos de almacenamiento y eficiencia espacial.  

No obstante, esta optimización no siempre se traduce en una reducción del uso de memoria durante la visualización. En muchos casos, la \textbf{asignación de memoria} de estas estructuras es prácticamente idéntica —e incluso puede ser ligeramente superior— a la de los datos originales, debido a los procesos internos de decodificación y representación que realizan las bibliotecas de visualización.  

Esto implica que, aunque ganamos en almacenamiento, no debemos esperar mejoras significativas en el consumo de RAM. Aun así, el uso adicional de memoria que puedan introducir estas estructuras compactas no es lo suficientemente alto como para tener un impacto relevante en la ejecución o limitar su uso en los experimentos.  

En bibliotecas como \textit{Pygal} y \textit{Vega-Altair}, el consumo de memoria al graficar datos procesados por \texttt{CompressedVectorDownsampler} resulta prácticamente igual al obtenido con datos reducidos mediante \textit{downsampling} estándar, lo que confirma que la principal ventaja de estas estructuras radica en el \textbf{ahorro de espacio en disco}, no en una optimización de memoria en tiempo de ejecución.




\chapter{Conclusiones}

El carácter de la Memoria de Título es principalmente experimental. Las herramientas desarrolladas durante su realización han servido para delimitar un esquema general del potencial respecto al uso de estructuras de datos compactas para la visualización de datos, centrándose en una visión general de los caminos posibles que se ramifican a partir de la investigación realizada y los resultados obtenidos.

El uso de estructuras de datos compactas para la visualización de datos es un campo en constante evolución, y los resultados obtenidos en esta Memoria de Título sugieren que hay un gran potencial para mejorar la eficiencia y la efectividad de las visualizaciones. A pesar de que éstas no han sido implementadas para su uso con valores flotantes, el desarrollo de la clase \texttt{CompressedVector} demuestra que es posible superar esta limitación y beneficiarse de la reducción del espacio en disco a la hora de renderizar gráficos.

Además, su uso integrado con la biblioteca de \textit{downsampling} \texttt{tsdownsample}, a través de la clase desarrollada \texttt{CompressedVectorDownsampler}, demuestra que es posible combinar técnicas de compresión y reducción de datos para visualizar grandes conjuntos de datos, obteniendo una mejora significativa en tiempo y espacio de almacenamiento, sin sacrificar la calidad de la visualización y la capacidad de análisis de los mismos.

Una limitación importante identificada es la asignación de memoria, que puede ser un factor limitante a la hora de ocupar la herramienta desarrollada. Si el usuario se viera perjudicado por un leve aumento de la misma, utilizar las herramientas desarrolladas supondría un problema o incluso una restricción para su uso.
La escasez de bibliotecas de visualización que soporten iterar por sobre objetos sin la necesidad de asignar memoria adicional, o preprocesar los datos para su visualización, es una problema que sólo es evidente al momento del desarrollo de este trabajo.

Por último, es importante destacar que la investigación y el desarrollo en este campo están en constante evolución. Las herramientas y técnicas desarrolladas en esta Memoria de Título son un primer paso hacia una comprensión más profunda de cómo las estructuras de datos compactas pueden mejorar la visualización de datos. Se espera que futuras investigaciones continúen explorando estas posibilidades y desarrollen nuevas técnicas y herramientas para mejorar aún más la eficiencia y efectividad de las visualizaciones.

\chapter{Proyecciones}
\label{proyecciones}
Hay muchas direcciones futuras que se pueden explorar a partir del trabajo realizado en esta Memoria de Título. Debido a su naturaleza general y experimental, son muy variadas las rutas por las cuales continuar la investigación y el desarrollo en este campo.

\section{Mejoras en la clase \texttt{CompressedVector}}

La implementación de la clase \texttt{CompressedVector} es un primer acercamiento que podría ser considerado un prototipo. A pesar de que cumple su función de reducir el espacio en disco contra los datos originales, aún existen muchos aspectos que podrían ser mejorados para garantizar una reducción mayor del espacio y una mejora en la velocidad de acceso a los datos.

\subsection{Reducir la cantidad de vectores \texttt{SDSL4Py} usados.}

Como se explicó anteriormente en la sección \ref{sec:development:compressed_vector}, la clase \texttt{CompressedVector} utiliza múltiples vectores de SDSL4Py para almacenar los datos. Esto provoca que por cada vector que usaría un espacio $s_{sdsl4py}$ en una estrucutura de SDSL4Py, la clase \texttt{CompressedVector} use un espacio $3s_{sdsl4py}$ en el peor caso. 

Este problema podría ser mitigado eliminando el vector interno \texttt{sign\_part}, que actualmente indica si cada valor es positivo o negativo. En su lugar, se podría aplicar una transformación previa sobre el vector original para asegurar que todos los valores sean no negativos. Una forma común de lograr esto es restar el valor mínimo del conjunto a todos los elementos y almacenar este desplazamiento (\textit{offset}) por separado. Esta técnica es conocida como \textit{offset encoding} y permite reducir el número de vectores requeridos internamente, mejorando el uso de memoria y el espacio usado en disco.

Esta mejora puede verse en el experimento\ref{comparison_of_space_used}, donde al usar el input de taxis en New York\ref{input_taxi} se ve una mejora considerable del uso del espacio, al no tener la necesidad de usar un vector para almacenar los decimales.

\begin{equation}
x_i' = x_i - \min(x)
\end{equation}

Donde \( x_i \) representa cada elemento original del vector, y \( x_i' \) el valor transformado. El valor mínimo \(\min(x)\) se almacena de forma externa y puede ser sumado nuevamente al momento de acceder a los datos originales.

También podría explorarse la idea de eliminar el vector interno \texttt{decimal\_part}, que almacena la parte decimal de cada elemento. En su lugar, se puede aplicar un escalamiento global a todos los valores originales multiplicándolos por \(10^d\) (donde \(d\) es la cantidad de decimales que se desea conservar) y redondeando el resultado a enteros:

\begin{equation}
\tilde{x}_i = \operatorname{round}\!\left( x_i \cdot 10^d \right)
\end{equation}

De esta forma, toda la información queda unificada en un único vector compacto de enteros, acompañado únicamente por el metadato \(d\) que indica la precisión utilizada. Este enfoque reduce el número de vectores internos y, por lo tanto, el espacio total en disco, a costa de una sobrecarga mínima para revertir la operación durante la reconstrucción de los datos originales.

\subsection{Utilizar ancho de elementos y compactación de forma dinámica.}

Una mejora futura ideal para este proyecto podría ser la capacidad de analizar el input y automáticamente aplicar la mejor estrategia de compactación, posiciones decimales, ancho de los elementos e incluso la técnica de downsampling.

\subsection{Optimizar el acceso secuencial y contiguo a los datos.}
\label{proyecciones_vlc}
La clase \texttt{CompressedVector} actualmente accede a los datos mediante una iteración directa sobre los vectores internos, sin considerar el patrón de acceso. Este enfoque puede ser ineficiente en casos donde los accesos se realizan de manera contigua o secuencial, ya que no aprovecha las propiedades de localidad que podrían optimizar el rendimiento.

En particular, las estructuras provistas por SDSL permiten el uso eficiente de operaciones como \texttt{rank} y \texttt{select}, que pueden ser explotadas para acelerar el acceso en escenarios donde se requiere iterar sobre rangos consecutivos de índices. Ya que para visualizar un gráfico se itera sobre todos los datos en orden, se podría implementar un método que aproveche estas operaciones para acceder a los datos de manera más eficiente, reduciendo el tiempo de acceso y mejorando el rendimiento general de la clase.

\section{Bibliotecas de Visualización}

Existen diversas bibliotecas de visualización que podrían beneficiarse de las optimizaciones propuestas en esta Memoria de Título. Por ejemplo, bibliotecas como Matplotlib, Seaborn o Plotly en Python, que son ampliamente utilizadas para la creación de gráficos y visualizaciones interactivas, podrían integrar técnicas de compresión y acceso eficiente a los datos para mejorar su rendimiento. Sin embargo, debido a que estas bibliotecas requieren un tipo especifico de vector para graficar, la implementación de la clase \texttt{CompressedVector} no es directamente compatible con ellas.

Un trabajo futuro podría centrarse en desarrollar adaptadores o extensiones para estas bibliotecas, permitiendo que utilicen la clase \texttt{CompressedVector} para la visualización sin la necesidad de asignar memoria adicional. Esto podría implicar la creación de clases envoltorio que implementen las interfaces requeridas por estas bibliotecas, permitiendo que los datos comprimidos sean utilizados directamente para la visualización.

Otro posible acercamiento a este problema podría incluir la creación de otro \texttt{dtype} para la biblioteca NumPy, que permita el uso de la clase \texttt{CompressedVector} como un tipo de dato nativo. Esto permitiría que las bibliotecas de visualización que dependen de NumPy puedan utilizar directamente los datos comprimidos sin necesidad de realizar conversiones adicionales.



%%%%% Referencias
\bibliographystyle{plain}
\bibliography{references}

%%%%% Apéndices
\appendix
\renewcommand{\chaptertitlename}{Anexo}
\section{Anexo}

\subsection{Placeholder}
\label{anexo_sdsl4py}

\newpage

\subsection{Placeholder2}
\subsubsection{Resumen de Experimentos}
\label{anexo_resumen-experimentos}

\begin{longtable}{|p{3.8cm}|p{3.8cm}|p{4.2cm}|p{3.8cm}|p{2cm}|}
\caption{Resumen de los experimentos ejecutados en el sistema (ordenados alfabéticamente)} \label{tab:resumen-experimentos} \\
\hline
\textbf{Título (Español)} & \textbf{Título (Inglés)} & \textbf{Parámetros de entrada} & \textbf{Métrica evaluada} & \textbf{Unidad} \\
\hline
\endfirsthead

\multicolumn{5}{c}%
{{\bfseries \tablename\ \thetable{} -- continuación de la página anterior}} \\
\hline
\textbf{Título (Español)} & \textbf{Título (Inglés)} & \textbf{Parámetros de entrada} & \textbf{Métrica evaluada} & \textbf{Unidad} \\
\hline
\endhead

\hline \multicolumn{5}{|r|}{{Continúa en la siguiente página...}} \\ \hline
\endfoot

\hline
\endlastfoot

\hyperref[exp:cvd-access-decimals]{Acceso en CompressedVector con Diferentes Decimales} 
& CVD Decimal Places Access Time Comparison 
& \begin{minipage}[t]{\linewidth}\vspace{0.2em}
\begin{itemize}[leftmargin=*, noitemsep]
  \item Número de decimales
  \item Método de compresión
\end{itemize}
\vspace{-0.2em}
\end{minipage}
& Tiempo de acceso promedio por índice 
& segundos \\
\hline

\hyperref[exp:altair-mem]{Asignación de Memoria con Vega-Altair} 
& Vega-Altair Memory Allocation 
& \begin{minipage}[t]{\linewidth}\vspace{0.2em}
\begin{itemize}[leftmargin=*, noitemsep]
  \item Tipo de entrada
  \item Método de compresión y reducción
  \item $n_{out}$
\end{itemize}
\vspace{-0.2em}
\end{minipage}
& Memoria alocada al renderizar 
& kilobytes \\
\hline

\hyperref[exp:all-libs-mem]{Asignación de Memoria en Todas las Librerías} 
& All Libraries Memory Allocation 
& \begin{minipage}[t]{\linewidth}\vspace{0.2em}
\begin{itemize}[leftmargin=*, noitemsep]
  \item Librería
  \item Datos procesados por \texttt{CompressedVectorDownsampler}
\end{itemize}
\vspace{-0.2em}
\end{minipage}
& Memoria alocada al graficar 
& kilobytes \\
\hline

\hyperref[exp:pygal-mem]{Asignación de Memoria con Pygal} 
& Pygal Memory Allocation 
& \begin{minipage}[t]{\linewidth}\vspace{0.2em}
\begin{itemize}[leftmargin=*, noitemsep]
  \item Tipo de entrada
\end{itemize}
\vspace{-0.2em}
\end{minipage}
& Memoria alocada al renderizar 
& kilobytes \\
\hline

\hyperref[exp:altair-total-time]{Tiempo Total de Construcción y Graficado con Altair} 
& Vega-Altair Plotting + Building Comparison 
& \begin{minipage}[t]{\linewidth}\vspace{0.2em}
\begin{itemize}[leftmargin=*, noitemsep]
  \item Tipo de entrada
  \item Método de compresión
\end{itemize}
\vspace{-0.2em}
\end{minipage}
& Tiempo total (preparación + renderizado) 
& segundos \\
\hline

\hyperref[exp:altair-time]{Tiempos de Graficado con Vega-Altair} 
& Vega-Altair Plot Time Comparison 
& \begin{minipage}[t]{\linewidth}\vspace{0.2em}
\begin{itemize}[leftmargin=*, noitemsep]
  \item Tipo de entrada
  \item Método de compresión y reducción
  \item $n_{out}$
\end{itemize}
\vspace{-0.2em}
\end{minipage}
& Tiempo de renderizado 
& segundos \\
\hline

\hyperref[exp:all-libs-time]{Comparación de Tiempos de Graficado en Todas las Librerías} 
& All Libraries Time Comparison 
& \begin{minipage}[t]{\linewidth}\vspace{0.2em}
\begin{itemize}[leftmargin=*, noitemsep]
  \item Librería
  \item Datos procesados por \texttt{CompressedVectorDownsampler}
\end{itemize}
\vspace{-0.2em}
\end{minipage}
& Tiempo de renderizado 
& segundos \\
\hline

\hyperref[exp:compression-time]{Comparación de Tiempo de Compresión} 
& Compression Time Comparison 
& \begin{minipage}[t]{\linewidth}\vspace{0.2em}
\begin{itemize}[leftmargin=*, noitemsep]
  \item Tipo de entrada
  \item Método de compresión
  \item Método de reducción
  \item $n_{out}$
\end{itemize}
\vspace{-0.2em}
\end{minipage}
& Tiempo de procesamiento previo a la visualización 
& segundos \\
\hline

\hyperref[exp:space-comparison]{Comparación de Espacio Usado por Representación} 
& Comparison of Space Used 
& \begin{minipage}[t]{\linewidth}\vspace{0.2em}
\begin{itemize}[leftmargin=*, noitemsep]
  \item Tipo de estructura de representación
\end{itemize}
\vspace{-0.2em}
\end{minipage}
& Tamaño en memoria 
& bytes \\
\hline

\hyperref[exp:build-cv-sdsl]{Construcción de CompressedVector vs Vector SDSL4PY} 
& Time for Building Compressed Vector and SDSL4PY Vector 
& \begin{minipage}[t]{\linewidth}\vspace{0.2em}
\begin{itemize}[leftmargin=*, noitemsep]
  \item Tipo de vector
  \item Método de compresión
\end{itemize}
\vspace{-0.2em}
\end{minipage}
& Tiempo de construcción 
& segundos \\
\hline

\hyperref[exp:cvd-build-decimals]{Construcción de CompressedVector con Diferentes Decimales} 
& CVD Decimal Places Build Time Comparison 
& \begin{minipage}[t]{\linewidth}\vspace{0.2em}
\begin{itemize}[leftmargin=*, noitemsep]
  \item Número de decimales
  \item Método de compresión
  \item Método de reducción
  \item $n_{out}$
\end{itemize}
\vspace{-0.2em}
\end{minipage}
& Tiempo de construcción 
& segundos \\
\hline

\hyperref[exp:build-cvd-sdsl]{Construcción de Vector desde Downsampling vs Vector SDSL4PY} 
& Time for Building Compressed Vector Downsampler and SDSL4PY Vector 
& \begin{minipage}[t]{\linewidth}\vspace{0.2em}
\begin{itemize}[leftmargin=*, noitemsep]
  \item Método de compresión
  \item Método de reducción
  \item $n_{out}$
\end{itemize}
\vspace{-0.2em}
\end{minipage}
& Tiempo de construcción 
& segundos \\
\hline

\hyperref[exp:cvd-size-decimals]{Tamaño en Memoria de CompressedVector con Diferentes Decimales} 
& CVD Decimal Places Size Comparison 
& \begin{minipage}[t]{\linewidth}\vspace{0.2em}
\begin{itemize}[leftmargin=*, noitemsep]
  \item Número de decimales
  \item Método de compresión
\end{itemize}
\vspace{-0.2em}
\end{minipage}
& Tamaño en memoria 
& bytes \\
\hline

\hyperref[exp:pygal-time]{Tiempos de Graficado con Pygal} 
& Pygal Plot Time Comparison 
& \begin{minipage}[t]{\linewidth}\vspace{0.2em}
\begin{itemize}[leftmargin=*, noitemsep]
  \item Tipo de entrada
\end{itemize}
\vspace{-0.2em}
\end{minipage}
& Tiempo de renderizado 
& segundos \\
\hline

\end{longtable}



\end{document}
