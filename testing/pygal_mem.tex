\subsubsection{Asignación de Memoria con Pygal}
\label{exp:pygal-mem}

% imagen
\begin{figure}[H]
    \centering
    \includegraphics[width=0.8\textwidth]{testing/images/pygal_memory_allocation.png}
    \caption{Resultados de los experimentos de asignación de memoria con Pygal.}
    \label{fig:pygal-memory-allocation}
\end{figure}

Los resultados del experimento demuestran que la memoria alocada por la visualización de los datos con la librería PyGal es proporcional al tamaño de los datos procesados. Por ello, se puede observar que al utilizar los datos procesados de \texttt{CompressedVectorDownsampler} se aloca menos memoria que los datos originales, e incluso que los datos procesados por \texttt{TSDownsampler}.
